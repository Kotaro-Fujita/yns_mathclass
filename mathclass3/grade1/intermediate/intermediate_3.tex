% Options for packages loaded elsewhere
\PassOptionsToPackage{unicode}{hyperref}
\PassOptionsToPackage{hyphens}{url}
%
\documentclass[
  12pt,a4paper,lualatex,ja=standard]{bxjsarticle}
\usepackage{lmodern}
\usepackage{amsmath}
\usepackage{ifxetex,ifluatex}
\ifnum 0\ifxetex 1\fi\ifluatex 1\fi=0 % if pdftex
  \usepackage[T1]{fontenc}
  \usepackage[utf8]{inputenc}
  \usepackage{textcomp} % provide euro and other symbols
  \usepackage{amssymb}
\else % if luatex or xetex
  \usepackage{unicode-math}
  \defaultfontfeatures{Scale=MatchLowercase}
  \defaultfontfeatures[\rmfamily]{Ligatures=TeX,Scale=1}
\fi
% Use upquote if available, for straight quotes in verbatim environments
\IfFileExists{upquote.sty}{\usepackage{upquote}}{}
\IfFileExists{microtype.sty}{% use microtype if available
  \usepackage[]{microtype}
  \UseMicrotypeSet[protrusion]{basicmath} % disable protrusion for tt fonts
}{}
\makeatletter
\@ifundefined{KOMAClassName}{% if non-KOMA class
  \IfFileExists{parskip.sty}{%
    \usepackage{parskip}
  }{% else
    \setlength{\parindent}{0pt}
    \setlength{\parskip}{6pt plus 2pt minus 1pt}}
}{% if KOMA class
  \KOMAoptions{parskip=half}}
\makeatother
\usepackage{xcolor}
\IfFileExists{xurl.sty}{\usepackage{xurl}}{} % add URL line breaks if available
\IfFileExists{bookmark.sty}{\usepackage{bookmark}}{\usepackage{hyperref}}
\hypersetup{
  hidelinks,
  pdfcreator={LaTeX via pandoc}}
\urlstyle{same} % disable monospaced font for URLs
\usepackage{graphicx}
\makeatletter
\def\maxwidth{\ifdim\Gin@nat@width>\linewidth\linewidth\else\Gin@nat@width\fi}
\def\maxheight{\ifdim\Gin@nat@height>\textheight\textheight\else\Gin@nat@height\fi}
\makeatother
% Scale images if necessary, so that they will not overflow the page
% margins by default, and it is still possible to overwrite the defaults
% using explicit options in \includegraphics[width, height, ...]{}
\setkeys{Gin}{width=\maxwidth,height=\maxheight,keepaspectratio}
% Set default figure placement to htbp
\makeatletter
\def\fps@figure{htbp}
\makeatother
\setlength{\emergencystretch}{3em} % prevent overfull lines
\providecommand{\tightlist}{%
  \setlength{\itemsep}{0pt}\setlength{\parskip}{0pt}}
\setcounter{secnumdepth}{5}
\usepackage{indentfirst}
\parindent = 1em
\usepackage{dcolumn}
\newcolumntype{.}{D{.}{.}{-1}}
\usepackage{caption}
\captionsetup[table]{name=表}
\captionsetup[figure]{name=図}
\usepackage{hyperref}
\pagestyle{empty}
\usepackage{multicol}
\usepackage{ascmac}
\setpagelayout*{top=10truemm,bottom=30truemm,left=10truemm,right=10truemm}
\usepackage{tikz}
\usetikzlibrary{arrows.meta,decorations,decorations.pathreplacing,arrows,calc}
\usepackage{tabstackengine}
\usepackage{xcolor}
\usepackage{rotating}
\usepackage{txfonts}
\usepackage{fancybox}
\usepackage{dashbox}
\usepackage{tcolorbox}
\tcbuselibrary{theorems,skins}
\usepackage{siunitx}
\usepackage{framed}
\usepackage{enumerate}
\usepackage{lastpage}
\usepackage{pgfplots}
\pgfplotsset{compat=1.15}
\usepackage{mathrsfs}
\ifluatex
  \usepackage{selnolig}  % disable illegal ligatures
\fi

\author{}
\date{\vspace{-2.5em}}

\begin{document}

\renewcommand{\thefootnote}{}
\newcounter{kaunta}
\renewcommand{\thekaunta}{\arabic{kaunta}}
\newcommand{\kaunta}{\refstepcounter{kaunta}%
\thekaunta}
\def\question{\noindent\fbox{\large\makebox[1em]{\text{\kaunta}}} \hspace{1pt}}
\newcounter{skaunta}
\renewcommand{\theskaunta}{\arabic{skaunta}}
\newcommand{\skaunta}{\refstepcounter{skaunta}%
\theskaunta}
\def\squestion{(\text{\skaunta})\hspace{2.5pt}}
\newcommand{\maru}[1]{\raise0.2ex\hbox{\textcircled{\scriptsize{#1}}}}
\newcommand{\jsim}{\mathrel{\text{∽}}}
\newcommand{\jpara}{/\!/}

\newcounter{kcounter}
\setcounter{kcounter}{0}
\newcommand{\kana}{\refstepcounter{kcounter}\ifthenelse{\value{kcounter}=1}{ア}{\ifthenelse{\value{kcounter}=2}{イ}{\ifthenelse{\value{kcounter}=3}{ウ}{\ifthenelse{\value{kcounter}=4}{エ}{\ifthenelse{\value{kcounter}=5}{オ} {\ifthenelse{\value{kcounter}=6}{カ}{\ifthenelse{\value{kcounter}=7}{キ}{\ifthenelse{\value{kcounter}=8}{ク}{\ifthenelse{\value{kcounter}=9}{ケ}{\ifthenelse{\value{kcounter}=10}{コ}{\ifthenelse{\value{kcounter}=11}{サ}{\ifthenelse{\value{kcounter}=12}{シ}{\ifthenelse{\value{kcounter}=13}{ス}{\ifthenelse{\value{kcounter}=14}{セ}{\ifthenelse{\value{kcounter}=15}{ソ}{\ifthenelse{\value{kcounter}=16}{タ}{\ifthenelse{\value{kcounter}=17}{チ}{\ifthenelse{\value{kcounter}=18}{ツ}{\ifthenelse{\value{kcounter}=19}{テ}{\ifthenelse{\value{kcounter}=20}{ト}{\ifthenelse{\value{kcounter}=21}{ナ}{\ifthenelse{\value{kcounter}=22}{ニ}{\ifthenelse{\value{kcounter}=23}{ヌ}{\ifthenelse{\value{kcounter}=24}{ネ}{\ifthenelse{\value{kcounter}=25}{ノ}{\ifthenelse{\value{kcounter}=26}{ハ}{\ifthenelse{\value{kcounter}=27}{ヒ}{\ifthenelse{\value{kcounter}=28}{フ}{\ifthenelse{\value{kcounter}=29}{ヘ}{\ifthenelse{\value{kcounter}=30}{ホ}{\ifthenelse{\value{kcounter}=31}{マ}{\ifthenelse{\value{kcounter}=32}{ミ}{\ifthenelse{\value{kcounter}=33}{ム}{\ifthenelse{\value{kcounter}=34}{メ}{\ifthenelse{\value{kcounter}=35}{モ}{\ifthenelse{\value{kcounter}=36}{ヤ}{\ifthenelse{\value{kcounter}=37}{ユ}{\ifthenelse{\value{kcounter}=38}{ヨ}{\ifthenelse{\value{kcounter}=39}{ラ}{\ifthenelse{\value{kcounter}=40}{リ}{\ifthenelse{\value{kcounter}=41}{ル}{\ifthenelse{\value{kcounter}=42}{レ}{\ifthenelse{\value{kcounter}=43}{ロ}{\ifthenelse{\value{kcounter}=44}{ワ}{・}}}}}}}}}}}}}}}}}}}}}}}}}}}}}}}}}}}}}}}}}}}}}

\newcommand{\kuran}[1]{\framebox[1.5cm][c]{\maru{\kana}}}

\newcommand{\degre}{\ensuremath{^\circ}}

\newcommand{\myarc}[1]{
   \tikz [baseline = (N.base), every node/.style={}] {
      \node [inner sep = 0pt] (N) {$\mathrm{#1}$};
      \draw [line width = 0.4pt] plot [smooth, tension=1.3] coordinates {
         ($(N.north west) + (0.1ex,0)$)
         ($(N.north)      + (0,0.5ex)$)
         ($(N.north east) + (0,0)$)
      };
   }
}

\makeatletter
\newenvironment{figurehere}{\def\@captype{figure}}{}
\makeatother

\newcommand{\haiten}[1]{%
\begin{flushright}%
\footnotesize{<#1>}%
\end{flushright}%
}
\newgeometry{top=10truemm,bottom=10truemm,left=20truemm,right=20truemm}

\thispagestyle{empty}
\begin{center}
\phantom{empty}

\vspace{60truemm}

\hspace{4em} {\HUGE\gtfamily\bfseries 数\hspace{2em}学}\hspace{1em}{\large \gtfamily \bfseries ($\mathbf{1}$年)}\\

\vspace{15truemm}

\hspace{2.5em}{\large \gtfamily \bfseries(この問題は定規とコンパスが必要です。)}

\vspace{64truemm}

{\large\gtfamily\bfseries 注\hspace{5em}意}

\end{center}

\centering
\begin{framed}
\begin{flushleft}
\begin{enumerate}[\Large \gtfamily 1]
  \item {\large 「開始」の合図があるまでは,開いてはいけません。}

  \item {\large 問題は\pageref{LastPage}ページまであります。}

  \item {\large 「開始」の合図があったら,まず,問題用紙・解答用紙に,組・番号と名前などを書きなさい。}

  \item {\large 答えは,すべて解答用紙に書きなさい。また、所定の欄に濃くはっきりと書きなさい。}

  \item {\large 「終了」の合図で,すぐ鉛筆をおき,解答用紙を裏返しにしなさい。}
\end{enumerate}
\end{flushleft}
\end{framed}

\vspace{14mm}

\begin{center}
{\large \underline{\hspace{30mm}組 \hspace{30mm}番 \hspace{15mm} 名前 \hspace{60mm}}}
\end{center}

\pagestyle{plain}
\pagenumbering{arabic}

\begin{flushleft}

\noindent\fbox{\large\makebox[1em]{\text{\refstepcounter{kaunta}%
\arabic{kaunta}}}} \hspace{1pt}次の空欄に当てはまる言葉や記号を書きなさい。

%
\begin{flushright}%
\footnotesize{<知・技$1 \times 31$点>}%
\end{flushright}%


2点A, Bを通り、両方にかぎりなくのびているものを\framebox[1.5cm][c]{\raise 0.2ex\hbox{\textcircled{\scriptsize{\refstepcounter{kcounter}\ifthenelse{\value{kcounter}=1}{ア}{\ifthenelse{\value{kcounter}=2}{イ}{\ifthenelse{\value{kcounter}=3}{ウ}{\ifthenelse{\value{kcounter}=4}{エ}{\ifthenelse{\value{kcounter}=5}{オ} {\ifthenelse{\value{kcounter}=6}{カ}{\ifthenelse{\value{kcounter}=7}{キ}{\ifthenelse{\value{kcounter}=8}{ク}{\ifthenelse{\value{kcounter}=9}{ケ}{\ifthenelse{\value{kcounter}=10}{コ}{\ifthenelse{\value{kcounter}=11}{サ}{\ifthenelse{\value{kcounter}=12}{シ}{\ifthenelse{\value{kcounter}=13}{ス}{\ifthenelse{\value{kcounter}=14}{セ}{\ifthenelse{\value{kcounter}=15}{ソ}{\ifthenelse{\value{kcounter}=16}{タ}{\ifthenelse{\value{kcounter}=17}{チ}{\ifthenelse{\value{kcounter}=18}{ツ}{\ifthenelse{\value{kcounter}=19}{テ}{\ifthenelse{\value{kcounter}=20}{ト}{\ifthenelse{\value{kcounter}=21}{ナ}{\ifthenelse{\value{kcounter}=22}{ニ}{\ifthenelse{\value{kcounter}=23}{ヌ}{\ifthenelse{\value{kcounter}=24}{ネ}{\ifthenelse{\value{kcounter}=25}{ノ}{\ifthenelse{\value{kcounter}=26}{ハ}{\ifthenelse{\value{kcounter}=27}{ヒ}{\ifthenelse{\value{kcounter}=28}{フ}{\ifthenelse{\value{kcounter}=29}{ヘ}{\ifthenelse{\value{kcounter}=30}{ホ}{\ifthenelse{\value{kcounter}=31}{マ}{\ifthenelse{\value{kcounter}=32}{ミ}{\ifthenelse{\value{kcounter}=33}{ム}{\ifthenelse{\value{kcounter}=34}{メ}{\ifthenelse{\value{kcounter}=35}{モ}{\ifthenelse{\value{kcounter}=36}{ヤ}{\ifthenelse{\value{kcounter}=37}{ユ}{\ifthenelse{\value{kcounter}=38}{ヨ}{\ifthenelse{\value{kcounter}=39}{ラ}{\ifthenelse{\value{kcounter}=40}{リ}{\ifthenelse{\value{kcounter}=41}{ル}{\ifthenelse{\value{kcounter}=42}{レ}{\ifthenelse{\value{kcounter}=43}{ロ}{\ifthenelse{\value{kcounter}=44}{ワ}{・}}}}}}}}}}}}}}}}}}}}}}}}}}}}}}}}}}}}}}}}}}}}}}}}という。直線ABのうち、AからBまでの部分を\framebox[1.5cm][c]{\raise 0.2ex\hbox{\textcircled{\scriptsize{\refstepcounter{kcounter}\ifthenelse{\value{kcounter}=1}{ア}{\ifthenelse{\value{kcounter}=2}{イ}{\ifthenelse{\value{kcounter}=3}{ウ}{\ifthenelse{\value{kcounter}=4}{エ}{\ifthenelse{\value{kcounter}=5}{オ} {\ifthenelse{\value{kcounter}=6}{カ}{\ifthenelse{\value{kcounter}=7}{キ}{\ifthenelse{\value{kcounter}=8}{ク}{\ifthenelse{\value{kcounter}=9}{ケ}{\ifthenelse{\value{kcounter}=10}{コ}{\ifthenelse{\value{kcounter}=11}{サ}{\ifthenelse{\value{kcounter}=12}{シ}{\ifthenelse{\value{kcounter}=13}{ス}{\ifthenelse{\value{kcounter}=14}{セ}{\ifthenelse{\value{kcounter}=15}{ソ}{\ifthenelse{\value{kcounter}=16}{タ}{\ifthenelse{\value{kcounter}=17}{チ}{\ifthenelse{\value{kcounter}=18}{ツ}{\ifthenelse{\value{kcounter}=19}{テ}{\ifthenelse{\value{kcounter}=20}{ト}{\ifthenelse{\value{kcounter}=21}{ナ}{\ifthenelse{\value{kcounter}=22}{ニ}{\ifthenelse{\value{kcounter}=23}{ヌ}{\ifthenelse{\value{kcounter}=24}{ネ}{\ifthenelse{\value{kcounter}=25}{ノ}{\ifthenelse{\value{kcounter}=26}{ハ}{\ifthenelse{\value{kcounter}=27}{ヒ}{\ifthenelse{\value{kcounter}=28}{フ}{\ifthenelse{\value{kcounter}=29}{ヘ}{\ifthenelse{\value{kcounter}=30}{ホ}{\ifthenelse{\value{kcounter}=31}{マ}{\ifthenelse{\value{kcounter}=32}{ミ}{\ifthenelse{\value{kcounter}=33}{ム}{\ifthenelse{\value{kcounter}=34}{メ}{\ifthenelse{\value{kcounter}=35}{モ}{\ifthenelse{\value{kcounter}=36}{ヤ}{\ifthenelse{\value{kcounter}=37}{ユ}{\ifthenelse{\value{kcounter}=38}{ヨ}{\ifthenelse{\value{kcounter}=39}{ラ}{\ifthenelse{\value{kcounter}=40}{リ}{\ifthenelse{\value{kcounter}=41}{ル}{\ifthenelse{\value{kcounter}=42}{レ}{\ifthenelse{\value{kcounter}=43}{ロ}{\ifthenelse{\value{kcounter}=44}{ワ}{・}}}}}}}}}}}}}}}}}}}}}}}}}}}}}}}}}}}}}}}}}}}}}}}}という。また、線分ABをBのほうへまっすぐにかぎりなくのばしたものを\framebox[1.5cm][c]{\raise 0.2ex\hbox{\textcircled{\scriptsize{\refstepcounter{kcounter}\ifthenelse{\value{kcounter}=1}{ア}{\ifthenelse{\value{kcounter}=2}{イ}{\ifthenelse{\value{kcounter}=3}{ウ}{\ifthenelse{\value{kcounter}=4}{エ}{\ifthenelse{\value{kcounter}=5}{オ} {\ifthenelse{\value{kcounter}=6}{カ}{\ifthenelse{\value{kcounter}=7}{キ}{\ifthenelse{\value{kcounter}=8}{ク}{\ifthenelse{\value{kcounter}=9}{ケ}{\ifthenelse{\value{kcounter}=10}{コ}{\ifthenelse{\value{kcounter}=11}{サ}{\ifthenelse{\value{kcounter}=12}{シ}{\ifthenelse{\value{kcounter}=13}{ス}{\ifthenelse{\value{kcounter}=14}{セ}{\ifthenelse{\value{kcounter}=15}{ソ}{\ifthenelse{\value{kcounter}=16}{タ}{\ifthenelse{\value{kcounter}=17}{チ}{\ifthenelse{\value{kcounter}=18}{ツ}{\ifthenelse{\value{kcounter}=19}{テ}{\ifthenelse{\value{kcounter}=20}{ト}{\ifthenelse{\value{kcounter}=21}{ナ}{\ifthenelse{\value{kcounter}=22}{ニ}{\ifthenelse{\value{kcounter}=23}{ヌ}{\ifthenelse{\value{kcounter}=24}{ネ}{\ifthenelse{\value{kcounter}=25}{ノ}{\ifthenelse{\value{kcounter}=26}{ハ}{\ifthenelse{\value{kcounter}=27}{ヒ}{\ifthenelse{\value{kcounter}=28}{フ}{\ifthenelse{\value{kcounter}=29}{ヘ}{\ifthenelse{\value{kcounter}=30}{ホ}{\ifthenelse{\value{kcounter}=31}{マ}{\ifthenelse{\value{kcounter}=32}{ミ}{\ifthenelse{\value{kcounter}=33}{ム}{\ifthenelse{\value{kcounter}=34}{メ}{\ifthenelse{\value{kcounter}=35}{モ}{\ifthenelse{\value{kcounter}=36}{ヤ}{\ifthenelse{\value{kcounter}=37}{ユ}{\ifthenelse{\value{kcounter}=38}{ヨ}{\ifthenelse{\value{kcounter}=39}{ラ}{\ifthenelse{\value{kcounter}=40}{リ}{\ifthenelse{\value{kcounter}=41}{ル}{\ifthenelse{\value{kcounter}=42}{レ}{\ifthenelse{\value{kcounter}=43}{ロ}{\ifthenelse{\value{kcounter}=44}{ワ}{・}}}}}}}}}}}}}}}}}}}}}}}}}}}}}}}}}}}}}}}}}}}}}}}}という。

図形を、一定の方向に、一定の距離だけ動かす移動を\framebox[1.5cm][c]{\raise 0.2ex\hbox{\textcircled{\scriptsize{\refstepcounter{kcounter}\ifthenelse{\value{kcounter}=1}{ア}{\ifthenelse{\value{kcounter}=2}{イ}{\ifthenelse{\value{kcounter}=3}{ウ}{\ifthenelse{\value{kcounter}=4}{エ}{\ifthenelse{\value{kcounter}=5}{オ} {\ifthenelse{\value{kcounter}=6}{カ}{\ifthenelse{\value{kcounter}=7}{キ}{\ifthenelse{\value{kcounter}=8}{ク}{\ifthenelse{\value{kcounter}=9}{ケ}{\ifthenelse{\value{kcounter}=10}{コ}{\ifthenelse{\value{kcounter}=11}{サ}{\ifthenelse{\value{kcounter}=12}{シ}{\ifthenelse{\value{kcounter}=13}{ス}{\ifthenelse{\value{kcounter}=14}{セ}{\ifthenelse{\value{kcounter}=15}{ソ}{\ifthenelse{\value{kcounter}=16}{タ}{\ifthenelse{\value{kcounter}=17}{チ}{\ifthenelse{\value{kcounter}=18}{ツ}{\ifthenelse{\value{kcounter}=19}{テ}{\ifthenelse{\value{kcounter}=20}{ト}{\ifthenelse{\value{kcounter}=21}{ナ}{\ifthenelse{\value{kcounter}=22}{ニ}{\ifthenelse{\value{kcounter}=23}{ヌ}{\ifthenelse{\value{kcounter}=24}{ネ}{\ifthenelse{\value{kcounter}=25}{ノ}{\ifthenelse{\value{kcounter}=26}{ハ}{\ifthenelse{\value{kcounter}=27}{ヒ}{\ifthenelse{\value{kcounter}=28}{フ}{\ifthenelse{\value{kcounter}=29}{ヘ}{\ifthenelse{\value{kcounter}=30}{ホ}{\ifthenelse{\value{kcounter}=31}{マ}{\ifthenelse{\value{kcounter}=32}{ミ}{\ifthenelse{\value{kcounter}=33}{ム}{\ifthenelse{\value{kcounter}=34}{メ}{\ifthenelse{\value{kcounter}=35}{モ}{\ifthenelse{\value{kcounter}=36}{ヤ}{\ifthenelse{\value{kcounter}=37}{ユ}{\ifthenelse{\value{kcounter}=38}{ヨ}{\ifthenelse{\value{kcounter}=39}{ラ}{\ifthenelse{\value{kcounter}=40}{リ}{\ifthenelse{\value{kcounter}=41}{ル}{\ifthenelse{\value{kcounter}=42}{レ}{\ifthenelse{\value{kcounter}=43}{ロ}{\ifthenelse{\value{kcounter}=44}{ワ}{・}}}}}}}}}}}}}}}}}}}}}}}}}}}}}}}}}}}}}}}}}}}}}}}}という。線分ABと線分CDの長さが等しいことをAB \framebox[1.5cm][c]{\raise 0.2ex\hbox{\textcircled{\scriptsize{\refstepcounter{kcounter}\ifthenelse{\value{kcounter}=1}{ア}{\ifthenelse{\value{kcounter}=2}{イ}{\ifthenelse{\value{kcounter}=3}{ウ}{\ifthenelse{\value{kcounter}=4}{エ}{\ifthenelse{\value{kcounter}=5}{オ} {\ifthenelse{\value{kcounter}=6}{カ}{\ifthenelse{\value{kcounter}=7}{キ}{\ifthenelse{\value{kcounter}=8}{ク}{\ifthenelse{\value{kcounter}=9}{ケ}{\ifthenelse{\value{kcounter}=10}{コ}{\ifthenelse{\value{kcounter}=11}{サ}{\ifthenelse{\value{kcounter}=12}{シ}{\ifthenelse{\value{kcounter}=13}{ス}{\ifthenelse{\value{kcounter}=14}{セ}{\ifthenelse{\value{kcounter}=15}{ソ}{\ifthenelse{\value{kcounter}=16}{タ}{\ifthenelse{\value{kcounter}=17}{チ}{\ifthenelse{\value{kcounter}=18}{ツ}{\ifthenelse{\value{kcounter}=19}{テ}{\ifthenelse{\value{kcounter}=20}{ト}{\ifthenelse{\value{kcounter}=21}{ナ}{\ifthenelse{\value{kcounter}=22}{ニ}{\ifthenelse{\value{kcounter}=23}{ヌ}{\ifthenelse{\value{kcounter}=24}{ネ}{\ifthenelse{\value{kcounter}=25}{ノ}{\ifthenelse{\value{kcounter}=26}{ハ}{\ifthenelse{\value{kcounter}=27}{ヒ}{\ifthenelse{\value{kcounter}=28}{フ}{\ifthenelse{\value{kcounter}=29}{ヘ}{\ifthenelse{\value{kcounter}=30}{ホ}{\ifthenelse{\value{kcounter}=31}{マ}{\ifthenelse{\value{kcounter}=32}{ミ}{\ifthenelse{\value{kcounter}=33}{ム}{\ifthenelse{\value{kcounter}=34}{メ}{\ifthenelse{\value{kcounter}=35}{モ}{\ifthenelse{\value{kcounter}=36}{ヤ}{\ifthenelse{\value{kcounter}=37}{ユ}{\ifthenelse{\value{kcounter}=38}{ヨ}{\ifthenelse{\value{kcounter}=39}{ラ}{\ifthenelse{\value{kcounter}=40}{リ}{\ifthenelse{\value{kcounter}=41}{ル}{\ifthenelse{\value{kcounter}=42}{レ}{\ifthenelse{\value{kcounter}=43}{ロ}{\ifthenelse{\value{kcounter}=44}{ワ}{・}}}}}}}}}}}}}}}}}}}}}}}}}}}}}}}}}}}}}}}}}}}}}}}} CDとかく。図形をある点を中心として、一定の角度だけ回転させる移動を\framebox[1.5cm][c]{\raise 0.2ex\hbox{\textcircled{\scriptsize{\refstepcounter{kcounter}\ifthenelse{\value{kcounter}=1}{ア}{\ifthenelse{\value{kcounter}=2}{イ}{\ifthenelse{\value{kcounter}=3}{ウ}{\ifthenelse{\value{kcounter}=4}{エ}{\ifthenelse{\value{kcounter}=5}{オ} {\ifthenelse{\value{kcounter}=6}{カ}{\ifthenelse{\value{kcounter}=7}{キ}{\ifthenelse{\value{kcounter}=8}{ク}{\ifthenelse{\value{kcounter}=9}{ケ}{\ifthenelse{\value{kcounter}=10}{コ}{\ifthenelse{\value{kcounter}=11}{サ}{\ifthenelse{\value{kcounter}=12}{シ}{\ifthenelse{\value{kcounter}=13}{ス}{\ifthenelse{\value{kcounter}=14}{セ}{\ifthenelse{\value{kcounter}=15}{ソ}{\ifthenelse{\value{kcounter}=16}{タ}{\ifthenelse{\value{kcounter}=17}{チ}{\ifthenelse{\value{kcounter}=18}{ツ}{\ifthenelse{\value{kcounter}=19}{テ}{\ifthenelse{\value{kcounter}=20}{ト}{\ifthenelse{\value{kcounter}=21}{ナ}{\ifthenelse{\value{kcounter}=22}{ニ}{\ifthenelse{\value{kcounter}=23}{ヌ}{\ifthenelse{\value{kcounter}=24}{ネ}{\ifthenelse{\value{kcounter}=25}{ノ}{\ifthenelse{\value{kcounter}=26}{ハ}{\ifthenelse{\value{kcounter}=27}{ヒ}{\ifthenelse{\value{kcounter}=28}{フ}{\ifthenelse{\value{kcounter}=29}{ヘ}{\ifthenelse{\value{kcounter}=30}{ホ}{\ifthenelse{\value{kcounter}=31}{マ}{\ifthenelse{\value{kcounter}=32}{ミ}{\ifthenelse{\value{kcounter}=33}{ム}{\ifthenelse{\value{kcounter}=34}{メ}{\ifthenelse{\value{kcounter}=35}{モ}{\ifthenelse{\value{kcounter}=36}{ヤ}{\ifthenelse{\value{kcounter}=37}{ユ}{\ifthenelse{\value{kcounter}=38}{ヨ}{\ifthenelse{\value{kcounter}=39}{ラ}{\ifthenelse{\value{kcounter}=40}{リ}{\ifthenelse{\value{kcounter}=41}{ル}{\ifthenelse{\value{kcounter}=42}{レ}{\ifthenelse{\value{kcounter}=43}{ロ}{\ifthenelse{\value{kcounter}=44}{ワ}{・}}}}}}}}}}}}}}}}}}}}}}}}}}}}}}}}}}}}}}}}}}}}}}}}といい、中心とする点を\framebox[1.5cm][c]{\raise 0.2ex\hbox{\textcircled{\scriptsize{\refstepcounter{kcounter}\ifthenelse{\value{kcounter}=1}{ア}{\ifthenelse{\value{kcounter}=2}{イ}{\ifthenelse{\value{kcounter}=3}{ウ}{\ifthenelse{\value{kcounter}=4}{エ}{\ifthenelse{\value{kcounter}=5}{オ} {\ifthenelse{\value{kcounter}=6}{カ}{\ifthenelse{\value{kcounter}=7}{キ}{\ifthenelse{\value{kcounter}=8}{ク}{\ifthenelse{\value{kcounter}=9}{ケ}{\ifthenelse{\value{kcounter}=10}{コ}{\ifthenelse{\value{kcounter}=11}{サ}{\ifthenelse{\value{kcounter}=12}{シ}{\ifthenelse{\value{kcounter}=13}{ス}{\ifthenelse{\value{kcounter}=14}{セ}{\ifthenelse{\value{kcounter}=15}{ソ}{\ifthenelse{\value{kcounter}=16}{タ}{\ifthenelse{\value{kcounter}=17}{チ}{\ifthenelse{\value{kcounter}=18}{ツ}{\ifthenelse{\value{kcounter}=19}{テ}{\ifthenelse{\value{kcounter}=20}{ト}{\ifthenelse{\value{kcounter}=21}{ナ}{\ifthenelse{\value{kcounter}=22}{ニ}{\ifthenelse{\value{kcounter}=23}{ヌ}{\ifthenelse{\value{kcounter}=24}{ネ}{\ifthenelse{\value{kcounter}=25}{ノ}{\ifthenelse{\value{kcounter}=26}{ハ}{\ifthenelse{\value{kcounter}=27}{ヒ}{\ifthenelse{\value{kcounter}=28}{フ}{\ifthenelse{\value{kcounter}=29}{ヘ}{\ifthenelse{\value{kcounter}=30}{ホ}{\ifthenelse{\value{kcounter}=31}{マ}{\ifthenelse{\value{kcounter}=32}{ミ}{\ifthenelse{\value{kcounter}=33}{ム}{\ifthenelse{\value{kcounter}=34}{メ}{\ifthenelse{\value{kcounter}=35}{モ}{\ifthenelse{\value{kcounter}=36}{ヤ}{\ifthenelse{\value{kcounter}=37}{ユ}{\ifthenelse{\value{kcounter}=38}{ヨ}{\ifthenelse{\value{kcounter}=39}{ラ}{\ifthenelse{\value{kcounter}=40}{リ}{\ifthenelse{\value{kcounter}=41}{ル}{\ifthenelse{\value{kcounter}=42}{レ}{\ifthenelse{\value{kcounter}=43}{ロ}{\ifthenelse{\value{kcounter}=44}{ワ}{・}}}}}}}}}}}}}}}}}}}}}}}}}}}}}}}}}}}}}}}}}}}}}}}}という。図形を$\ang{180}$だけ回転移動させることを\framebox[1.5cm][c]{\raise 0.2ex\hbox{\textcircled{\scriptsize{\refstepcounter{kcounter}\ifthenelse{\value{kcounter}=1}{ア}{\ifthenelse{\value{kcounter}=2}{イ}{\ifthenelse{\value{kcounter}=3}{ウ}{\ifthenelse{\value{kcounter}=4}{エ}{\ifthenelse{\value{kcounter}=5}{オ} {\ifthenelse{\value{kcounter}=6}{カ}{\ifthenelse{\value{kcounter}=7}{キ}{\ifthenelse{\value{kcounter}=8}{ク}{\ifthenelse{\value{kcounter}=9}{ケ}{\ifthenelse{\value{kcounter}=10}{コ}{\ifthenelse{\value{kcounter}=11}{サ}{\ifthenelse{\value{kcounter}=12}{シ}{\ifthenelse{\value{kcounter}=13}{ス}{\ifthenelse{\value{kcounter}=14}{セ}{\ifthenelse{\value{kcounter}=15}{ソ}{\ifthenelse{\value{kcounter}=16}{タ}{\ifthenelse{\value{kcounter}=17}{チ}{\ifthenelse{\value{kcounter}=18}{ツ}{\ifthenelse{\value{kcounter}=19}{テ}{\ifthenelse{\value{kcounter}=20}{ト}{\ifthenelse{\value{kcounter}=21}{ナ}{\ifthenelse{\value{kcounter}=22}{ニ}{\ifthenelse{\value{kcounter}=23}{ヌ}{\ifthenelse{\value{kcounter}=24}{ネ}{\ifthenelse{\value{kcounter}=25}{ノ}{\ifthenelse{\value{kcounter}=26}{ハ}{\ifthenelse{\value{kcounter}=27}{ヒ}{\ifthenelse{\value{kcounter}=28}{フ}{\ifthenelse{\value{kcounter}=29}{ヘ}{\ifthenelse{\value{kcounter}=30}{ホ}{\ifthenelse{\value{kcounter}=31}{マ}{\ifthenelse{\value{kcounter}=32}{ミ}{\ifthenelse{\value{kcounter}=33}{ム}{\ifthenelse{\value{kcounter}=34}{メ}{\ifthenelse{\value{kcounter}=35}{モ}{\ifthenelse{\value{kcounter}=36}{ヤ}{\ifthenelse{\value{kcounter}=37}{ユ}{\ifthenelse{\value{kcounter}=38}{ヨ}{\ifthenelse{\value{kcounter}=39}{ラ}{\ifthenelse{\value{kcounter}=40}{リ}{\ifthenelse{\value{kcounter}=41}{ル}{\ifthenelse{\value{kcounter}=42}{レ}{\ifthenelse{\value{kcounter}=43}{ロ}{\ifthenelse{\value{kcounter}=44}{ワ}{・}}}}}}}}}}}}}}}}}}}}}}}}}}}}}}}}}}}}}}}}}}}}}}}}という。また、図形をある直線を折り目として折り返す移動を\framebox[1.5cm][c]{\raise 0.2ex\hbox{\textcircled{\scriptsize{\refstepcounter{kcounter}\ifthenelse{\value{kcounter}=1}{ア}{\ifthenelse{\value{kcounter}=2}{イ}{\ifthenelse{\value{kcounter}=3}{ウ}{\ifthenelse{\value{kcounter}=4}{エ}{\ifthenelse{\value{kcounter}=5}{オ} {\ifthenelse{\value{kcounter}=6}{カ}{\ifthenelse{\value{kcounter}=7}{キ}{\ifthenelse{\value{kcounter}=8}{ク}{\ifthenelse{\value{kcounter}=9}{ケ}{\ifthenelse{\value{kcounter}=10}{コ}{\ifthenelse{\value{kcounter}=11}{サ}{\ifthenelse{\value{kcounter}=12}{シ}{\ifthenelse{\value{kcounter}=13}{ス}{\ifthenelse{\value{kcounter}=14}{セ}{\ifthenelse{\value{kcounter}=15}{ソ}{\ifthenelse{\value{kcounter}=16}{タ}{\ifthenelse{\value{kcounter}=17}{チ}{\ifthenelse{\value{kcounter}=18}{ツ}{\ifthenelse{\value{kcounter}=19}{テ}{\ifthenelse{\value{kcounter}=20}{ト}{\ifthenelse{\value{kcounter}=21}{ナ}{\ifthenelse{\value{kcounter}=22}{ニ}{\ifthenelse{\value{kcounter}=23}{ヌ}{\ifthenelse{\value{kcounter}=24}{ネ}{\ifthenelse{\value{kcounter}=25}{ノ}{\ifthenelse{\value{kcounter}=26}{ハ}{\ifthenelse{\value{kcounter}=27}{ヒ}{\ifthenelse{\value{kcounter}=28}{フ}{\ifthenelse{\value{kcounter}=29}{ヘ}{\ifthenelse{\value{kcounter}=30}{ホ}{\ifthenelse{\value{kcounter}=31}{マ}{\ifthenelse{\value{kcounter}=32}{ミ}{\ifthenelse{\value{kcounter}=33}{ム}{\ifthenelse{\value{kcounter}=34}{メ}{\ifthenelse{\value{kcounter}=35}{モ}{\ifthenelse{\value{kcounter}=36}{ヤ}{\ifthenelse{\value{kcounter}=37}{ユ}{\ifthenelse{\value{kcounter}=38}{ヨ}{\ifthenelse{\value{kcounter}=39}{ラ}{\ifthenelse{\value{kcounter}=40}{リ}{\ifthenelse{\value{kcounter}=41}{ル}{\ifthenelse{\value{kcounter}=42}{レ}{\ifthenelse{\value{kcounter}=43}{ロ}{\ifthenelse{\value{kcounter}=44}{ワ}{・}}}}}}}}}}}}}}}}}}}}}}}}}}}}}}}}}}}}}}}}}}}}}}}}といい、折り目の直線を\framebox[1.5cm][c]{\raise 0.2ex\hbox{\textcircled{\scriptsize{\refstepcounter{kcounter}\ifthenelse{\value{kcounter}=1}{ア}{\ifthenelse{\value{kcounter}=2}{イ}{\ifthenelse{\value{kcounter}=3}{ウ}{\ifthenelse{\value{kcounter}=4}{エ}{\ifthenelse{\value{kcounter}=5}{オ} {\ifthenelse{\value{kcounter}=6}{カ}{\ifthenelse{\value{kcounter}=7}{キ}{\ifthenelse{\value{kcounter}=8}{ク}{\ifthenelse{\value{kcounter}=9}{ケ}{\ifthenelse{\value{kcounter}=10}{コ}{\ifthenelse{\value{kcounter}=11}{サ}{\ifthenelse{\value{kcounter}=12}{シ}{\ifthenelse{\value{kcounter}=13}{ス}{\ifthenelse{\value{kcounter}=14}{セ}{\ifthenelse{\value{kcounter}=15}{ソ}{\ifthenelse{\value{kcounter}=16}{タ}{\ifthenelse{\value{kcounter}=17}{チ}{\ifthenelse{\value{kcounter}=18}{ツ}{\ifthenelse{\value{kcounter}=19}{テ}{\ifthenelse{\value{kcounter}=20}{ト}{\ifthenelse{\value{kcounter}=21}{ナ}{\ifthenelse{\value{kcounter}=22}{ニ}{\ifthenelse{\value{kcounter}=23}{ヌ}{\ifthenelse{\value{kcounter}=24}{ネ}{\ifthenelse{\value{kcounter}=25}{ノ}{\ifthenelse{\value{kcounter}=26}{ハ}{\ifthenelse{\value{kcounter}=27}{ヒ}{\ifthenelse{\value{kcounter}=28}{フ}{\ifthenelse{\value{kcounter}=29}{ヘ}{\ifthenelse{\value{kcounter}=30}{ホ}{\ifthenelse{\value{kcounter}=31}{マ}{\ifthenelse{\value{kcounter}=32}{ミ}{\ifthenelse{\value{kcounter}=33}{ム}{\ifthenelse{\value{kcounter}=34}{メ}{\ifthenelse{\value{kcounter}=35}{モ}{\ifthenelse{\value{kcounter}=36}{ヤ}{\ifthenelse{\value{kcounter}=37}{ユ}{\ifthenelse{\value{kcounter}=38}{ヨ}{\ifthenelse{\value{kcounter}=39}{ラ}{\ifthenelse{\value{kcounter}=40}{リ}{\ifthenelse{\value{kcounter}=41}{ル}{\ifthenelse{\value{kcounter}=42}{レ}{\ifthenelse{\value{kcounter}=43}{ロ}{\ifthenelse{\value{kcounter}=44}{ワ}{・}}}}}}}}}}}}}}}}}}}}}}}}}}}}}}}}}}}}}}}}}}}}}}}}という。

平行な2直線を\framebox[1.5cm][c]{\raise 0.2ex\hbox{\textcircled{\scriptsize{\refstepcounter{kcounter}\ifthenelse{\value{kcounter}=1}{ア}{\ifthenelse{\value{kcounter}=2}{イ}{\ifthenelse{\value{kcounter}=3}{ウ}{\ifthenelse{\value{kcounter}=4}{エ}{\ifthenelse{\value{kcounter}=5}{オ} {\ifthenelse{\value{kcounter}=6}{カ}{\ifthenelse{\value{kcounter}=7}{キ}{\ifthenelse{\value{kcounter}=8}{ク}{\ifthenelse{\value{kcounter}=9}{ケ}{\ifthenelse{\value{kcounter}=10}{コ}{\ifthenelse{\value{kcounter}=11}{サ}{\ifthenelse{\value{kcounter}=12}{シ}{\ifthenelse{\value{kcounter}=13}{ス}{\ifthenelse{\value{kcounter}=14}{セ}{\ifthenelse{\value{kcounter}=15}{ソ}{\ifthenelse{\value{kcounter}=16}{タ}{\ifthenelse{\value{kcounter}=17}{チ}{\ifthenelse{\value{kcounter}=18}{ツ}{\ifthenelse{\value{kcounter}=19}{テ}{\ifthenelse{\value{kcounter}=20}{ト}{\ifthenelse{\value{kcounter}=21}{ナ}{\ifthenelse{\value{kcounter}=22}{ニ}{\ifthenelse{\value{kcounter}=23}{ヌ}{\ifthenelse{\value{kcounter}=24}{ネ}{\ifthenelse{\value{kcounter}=25}{ノ}{\ifthenelse{\value{kcounter}=26}{ハ}{\ifthenelse{\value{kcounter}=27}{ヒ}{\ifthenelse{\value{kcounter}=28}{フ}{\ifthenelse{\value{kcounter}=29}{ヘ}{\ifthenelse{\value{kcounter}=30}{ホ}{\ifthenelse{\value{kcounter}=31}{マ}{\ifthenelse{\value{kcounter}=32}{ミ}{\ifthenelse{\value{kcounter}=33}{ム}{\ifthenelse{\value{kcounter}=34}{メ}{\ifthenelse{\value{kcounter}=35}{モ}{\ifthenelse{\value{kcounter}=36}{ヤ}{\ifthenelse{\value{kcounter}=37}{ユ}{\ifthenelse{\value{kcounter}=38}{ヨ}{\ifthenelse{\value{kcounter}=39}{ラ}{\ifthenelse{\value{kcounter}=40}{リ}{\ifthenelse{\value{kcounter}=41}{ル}{\ifthenelse{\value{kcounter}=42}{レ}{\ifthenelse{\value{kcounter}=43}{ロ}{\ifthenelse{\value{kcounter}=44}{ワ}{・}}}}}}}}}}}}}}}}}}}}}}}}}}}}}}}}}}}}}}}}}}}}}}}}という。2直線AB, CDが平行であることを

\noindent
AB \framebox[1.5cm][c]{\raise 0.2ex\hbox{\textcircled{\scriptsize{\refstepcounter{kcounter}\ifthenelse{\value{kcounter}=1}{ア}{\ifthenelse{\value{kcounter}=2}{イ}{\ifthenelse{\value{kcounter}=3}{ウ}{\ifthenelse{\value{kcounter}=4}{エ}{\ifthenelse{\value{kcounter}=5}{オ} {\ifthenelse{\value{kcounter}=6}{カ}{\ifthenelse{\value{kcounter}=7}{キ}{\ifthenelse{\value{kcounter}=8}{ク}{\ifthenelse{\value{kcounter}=9}{ケ}{\ifthenelse{\value{kcounter}=10}{コ}{\ifthenelse{\value{kcounter}=11}{サ}{\ifthenelse{\value{kcounter}=12}{シ}{\ifthenelse{\value{kcounter}=13}{ス}{\ifthenelse{\value{kcounter}=14}{セ}{\ifthenelse{\value{kcounter}=15}{ソ}{\ifthenelse{\value{kcounter}=16}{タ}{\ifthenelse{\value{kcounter}=17}{チ}{\ifthenelse{\value{kcounter}=18}{ツ}{\ifthenelse{\value{kcounter}=19}{テ}{\ifthenelse{\value{kcounter}=20}{ト}{\ifthenelse{\value{kcounter}=21}{ナ}{\ifthenelse{\value{kcounter}=22}{ニ}{\ifthenelse{\value{kcounter}=23}{ヌ}{\ifthenelse{\value{kcounter}=24}{ネ}{\ifthenelse{\value{kcounter}=25}{ノ}{\ifthenelse{\value{kcounter}=26}{ハ}{\ifthenelse{\value{kcounter}=27}{ヒ}{\ifthenelse{\value{kcounter}=28}{フ}{\ifthenelse{\value{kcounter}=29}{ヘ}{\ifthenelse{\value{kcounter}=30}{ホ}{\ifthenelse{\value{kcounter}=31}{マ}{\ifthenelse{\value{kcounter}=32}{ミ}{\ifthenelse{\value{kcounter}=33}{ム}{\ifthenelse{\value{kcounter}=34}{メ}{\ifthenelse{\value{kcounter}=35}{モ}{\ifthenelse{\value{kcounter}=36}{ヤ}{\ifthenelse{\value{kcounter}=37}{ユ}{\ifthenelse{\value{kcounter}=38}{ヨ}{\ifthenelse{\value{kcounter}=39}{ラ}{\ifthenelse{\value{kcounter}=40}{リ}{\ifthenelse{\value{kcounter}=41}{ル}{\ifthenelse{\value{kcounter}=42}{レ}{\ifthenelse{\value{kcounter}=43}{ロ}{\ifthenelse{\value{kcounter}=44}{ワ}{・}}}}}}}}}}}}}}}}}}}}}}}}}}}}}}}}}}}}}}}}}}}}}}}} CDとかく。1つの点Oから出る2つの半直線OA, OBによって角ができる。この角を\framebox[1.5cm][c]{\raise 0.2ex\hbox{\textcircled{\scriptsize{\refstepcounter{kcounter}\ifthenelse{\value{kcounter}=1}{ア}{\ifthenelse{\value{kcounter}=2}{イ}{\ifthenelse{\value{kcounter}=3}{ウ}{\ifthenelse{\value{kcounter}=4}{エ}{\ifthenelse{\value{kcounter}=5}{オ} {\ifthenelse{\value{kcounter}=6}{カ}{\ifthenelse{\value{kcounter}=7}{キ}{\ifthenelse{\value{kcounter}=8}{ク}{\ifthenelse{\value{kcounter}=9}{ケ}{\ifthenelse{\value{kcounter}=10}{コ}{\ifthenelse{\value{kcounter}=11}{サ}{\ifthenelse{\value{kcounter}=12}{シ}{\ifthenelse{\value{kcounter}=13}{ス}{\ifthenelse{\value{kcounter}=14}{セ}{\ifthenelse{\value{kcounter}=15}{ソ}{\ifthenelse{\value{kcounter}=16}{タ}{\ifthenelse{\value{kcounter}=17}{チ}{\ifthenelse{\value{kcounter}=18}{ツ}{\ifthenelse{\value{kcounter}=19}{テ}{\ifthenelse{\value{kcounter}=20}{ト}{\ifthenelse{\value{kcounter}=21}{ナ}{\ifthenelse{\value{kcounter}=22}{ニ}{\ifthenelse{\value{kcounter}=23}{ヌ}{\ifthenelse{\value{kcounter}=24}{ネ}{\ifthenelse{\value{kcounter}=25}{ノ}{\ifthenelse{\value{kcounter}=26}{ハ}{\ifthenelse{\value{kcounter}=27}{ヒ}{\ifthenelse{\value{kcounter}=28}{フ}{\ifthenelse{\value{kcounter}=29}{ヘ}{\ifthenelse{\value{kcounter}=30}{ホ}{\ifthenelse{\value{kcounter}=31}{マ}{\ifthenelse{\value{kcounter}=32}{ミ}{\ifthenelse{\value{kcounter}=33}{ム}{\ifthenelse{\value{kcounter}=34}{メ}{\ifthenelse{\value{kcounter}=35}{モ}{\ifthenelse{\value{kcounter}=36}{ヤ}{\ifthenelse{\value{kcounter}=37}{ユ}{\ifthenelse{\value{kcounter}=38}{ヨ}{\ifthenelse{\value{kcounter}=39}{ラ}{\ifthenelse{\value{kcounter}=40}{リ}{\ifthenelse{\value{kcounter}=41}{ル}{\ifthenelse{\value{kcounter}=42}{レ}{\ifthenelse{\value{kcounter}=43}{ロ}{\ifthenelse{\value{kcounter}=44}{ワ}{・}}}}}}}}}}}}}}}}}}}}}}}}}}}}}}}}}}}}}}}}}}}}}}}}AOBとかく。2直線が垂直であるとき、一方の直線を他方の直線の\framebox[1.5cm][c]{\raise 0.2ex\hbox{\textcircled{\scriptsize{\refstepcounter{kcounter}\ifthenelse{\value{kcounter}=1}{ア}{\ifthenelse{\value{kcounter}=2}{イ}{\ifthenelse{\value{kcounter}=3}{ウ}{\ifthenelse{\value{kcounter}=4}{エ}{\ifthenelse{\value{kcounter}=5}{オ} {\ifthenelse{\value{kcounter}=6}{カ}{\ifthenelse{\value{kcounter}=7}{キ}{\ifthenelse{\value{kcounter}=8}{ク}{\ifthenelse{\value{kcounter}=9}{ケ}{\ifthenelse{\value{kcounter}=10}{コ}{\ifthenelse{\value{kcounter}=11}{サ}{\ifthenelse{\value{kcounter}=12}{シ}{\ifthenelse{\value{kcounter}=13}{ス}{\ifthenelse{\value{kcounter}=14}{セ}{\ifthenelse{\value{kcounter}=15}{ソ}{\ifthenelse{\value{kcounter}=16}{タ}{\ifthenelse{\value{kcounter}=17}{チ}{\ifthenelse{\value{kcounter}=18}{ツ}{\ifthenelse{\value{kcounter}=19}{テ}{\ifthenelse{\value{kcounter}=20}{ト}{\ifthenelse{\value{kcounter}=21}{ナ}{\ifthenelse{\value{kcounter}=22}{ニ}{\ifthenelse{\value{kcounter}=23}{ヌ}{\ifthenelse{\value{kcounter}=24}{ネ}{\ifthenelse{\value{kcounter}=25}{ノ}{\ifthenelse{\value{kcounter}=26}{ハ}{\ifthenelse{\value{kcounter}=27}{ヒ}{\ifthenelse{\value{kcounter}=28}{フ}{\ifthenelse{\value{kcounter}=29}{ヘ}{\ifthenelse{\value{kcounter}=30}{ホ}{\ifthenelse{\value{kcounter}=31}{マ}{\ifthenelse{\value{kcounter}=32}{ミ}{\ifthenelse{\value{kcounter}=33}{ム}{\ifthenelse{\value{kcounter}=34}{メ}{\ifthenelse{\value{kcounter}=35}{モ}{\ifthenelse{\value{kcounter}=36}{ヤ}{\ifthenelse{\value{kcounter}=37}{ユ}{\ifthenelse{\value{kcounter}=38}{ヨ}{\ifthenelse{\value{kcounter}=39}{ラ}{\ifthenelse{\value{kcounter}=40}{リ}{\ifthenelse{\value{kcounter}=41}{ル}{\ifthenelse{\value{kcounter}=42}{レ}{\ifthenelse{\value{kcounter}=43}{ロ}{\ifthenelse{\value{kcounter}=44}{ワ}{・}}}}}}}}}}}}}}}}}}}}}}}}}}}}}}}}}}}}}}}}}}}}}}}}という。線分ABと直線$l$が垂直であることを、AB\framebox[1.5cm][c]{\raise 0.2ex\hbox{\textcircled{\scriptsize{\refstepcounter{kcounter}\ifthenelse{\value{kcounter}=1}{ア}{\ifthenelse{\value{kcounter}=2}{イ}{\ifthenelse{\value{kcounter}=3}{ウ}{\ifthenelse{\value{kcounter}=4}{エ}{\ifthenelse{\value{kcounter}=5}{オ} {\ifthenelse{\value{kcounter}=6}{カ}{\ifthenelse{\value{kcounter}=7}{キ}{\ifthenelse{\value{kcounter}=8}{ク}{\ifthenelse{\value{kcounter}=9}{ケ}{\ifthenelse{\value{kcounter}=10}{コ}{\ifthenelse{\value{kcounter}=11}{サ}{\ifthenelse{\value{kcounter}=12}{シ}{\ifthenelse{\value{kcounter}=13}{ス}{\ifthenelse{\value{kcounter}=14}{セ}{\ifthenelse{\value{kcounter}=15}{ソ}{\ifthenelse{\value{kcounter}=16}{タ}{\ifthenelse{\value{kcounter}=17}{チ}{\ifthenelse{\value{kcounter}=18}{ツ}{\ifthenelse{\value{kcounter}=19}{テ}{\ifthenelse{\value{kcounter}=20}{ト}{\ifthenelse{\value{kcounter}=21}{ナ}{\ifthenelse{\value{kcounter}=22}{ニ}{\ifthenelse{\value{kcounter}=23}{ヌ}{\ifthenelse{\value{kcounter}=24}{ネ}{\ifthenelse{\value{kcounter}=25}{ノ}{\ifthenelse{\value{kcounter}=26}{ハ}{\ifthenelse{\value{kcounter}=27}{ヒ}{\ifthenelse{\value{kcounter}=28}{フ}{\ifthenelse{\value{kcounter}=29}{ヘ}{\ifthenelse{\value{kcounter}=30}{ホ}{\ifthenelse{\value{kcounter}=31}{マ}{\ifthenelse{\value{kcounter}=32}{ミ}{\ifthenelse{\value{kcounter}=33}{ム}{\ifthenelse{\value{kcounter}=34}{メ}{\ifthenelse{\value{kcounter}=35}{モ}{\ifthenelse{\value{kcounter}=36}{ヤ}{\ifthenelse{\value{kcounter}=37}{ユ}{\ifthenelse{\value{kcounter}=38}{ヨ}{\ifthenelse{\value{kcounter}=39}{ラ}{\ifthenelse{\value{kcounter}=40}{リ}{\ifthenelse{\value{kcounter}=41}{ル}{\ifthenelse{\value{kcounter}=42}{レ}{\ifthenelse{\value{kcounter}=43}{ロ}{\ifthenelse{\value{kcounter}=44}{ワ}{・}}}}}}}}}}}}}}}}}}}}}}}}}}}}}}}}}}}}}}}}}}}}}}}}$l$とかく。

線分を2等分する点を、その線分の\framebox[1.5cm][c]{\raise 0.2ex\hbox{\textcircled{\scriptsize{\refstepcounter{kcounter}\ifthenelse{\value{kcounter}=1}{ア}{\ifthenelse{\value{kcounter}=2}{イ}{\ifthenelse{\value{kcounter}=3}{ウ}{\ifthenelse{\value{kcounter}=4}{エ}{\ifthenelse{\value{kcounter}=5}{オ} {\ifthenelse{\value{kcounter}=6}{カ}{\ifthenelse{\value{kcounter}=7}{キ}{\ifthenelse{\value{kcounter}=8}{ク}{\ifthenelse{\value{kcounter}=9}{ケ}{\ifthenelse{\value{kcounter}=10}{コ}{\ifthenelse{\value{kcounter}=11}{サ}{\ifthenelse{\value{kcounter}=12}{シ}{\ifthenelse{\value{kcounter}=13}{ス}{\ifthenelse{\value{kcounter}=14}{セ}{\ifthenelse{\value{kcounter}=15}{ソ}{\ifthenelse{\value{kcounter}=16}{タ}{\ifthenelse{\value{kcounter}=17}{チ}{\ifthenelse{\value{kcounter}=18}{ツ}{\ifthenelse{\value{kcounter}=19}{テ}{\ifthenelse{\value{kcounter}=20}{ト}{\ifthenelse{\value{kcounter}=21}{ナ}{\ifthenelse{\value{kcounter}=22}{ニ}{\ifthenelse{\value{kcounter}=23}{ヌ}{\ifthenelse{\value{kcounter}=24}{ネ}{\ifthenelse{\value{kcounter}=25}{ノ}{\ifthenelse{\value{kcounter}=26}{ハ}{\ifthenelse{\value{kcounter}=27}{ヒ}{\ifthenelse{\value{kcounter}=28}{フ}{\ifthenelse{\value{kcounter}=29}{ヘ}{\ifthenelse{\value{kcounter}=30}{ホ}{\ifthenelse{\value{kcounter}=31}{マ}{\ifthenelse{\value{kcounter}=32}{ミ}{\ifthenelse{\value{kcounter}=33}{ム}{\ifthenelse{\value{kcounter}=34}{メ}{\ifthenelse{\value{kcounter}=35}{モ}{\ifthenelse{\value{kcounter}=36}{ヤ}{\ifthenelse{\value{kcounter}=37}{ユ}{\ifthenelse{\value{kcounter}=38}{ヨ}{\ifthenelse{\value{kcounter}=39}{ラ}{\ifthenelse{\value{kcounter}=40}{リ}{\ifthenelse{\value{kcounter}=41}{ル}{\ifthenelse{\value{kcounter}=42}{レ}{\ifthenelse{\value{kcounter}=43}{ロ}{\ifthenelse{\value{kcounter}=44}{ワ}{・}}}}}}}}}}}}}}}}}}}}}}}}}}}}}}}}}}}}}}}}}}}}}}}}という。線分の中点を通り、その線分に垂直な直線をその線分の\framebox[1.5cm][c]{\raise 0.2ex\hbox{\textcircled{\scriptsize{\refstepcounter{kcounter}\ifthenelse{\value{kcounter}=1}{ア}{\ifthenelse{\value{kcounter}=2}{イ}{\ifthenelse{\value{kcounter}=3}{ウ}{\ifthenelse{\value{kcounter}=4}{エ}{\ifthenelse{\value{kcounter}=5}{オ} {\ifthenelse{\value{kcounter}=6}{カ}{\ifthenelse{\value{kcounter}=7}{キ}{\ifthenelse{\value{kcounter}=8}{ク}{\ifthenelse{\value{kcounter}=9}{ケ}{\ifthenelse{\value{kcounter}=10}{コ}{\ifthenelse{\value{kcounter}=11}{サ}{\ifthenelse{\value{kcounter}=12}{シ}{\ifthenelse{\value{kcounter}=13}{ス}{\ifthenelse{\value{kcounter}=14}{セ}{\ifthenelse{\value{kcounter}=15}{ソ}{\ifthenelse{\value{kcounter}=16}{タ}{\ifthenelse{\value{kcounter}=17}{チ}{\ifthenelse{\value{kcounter}=18}{ツ}{\ifthenelse{\value{kcounter}=19}{テ}{\ifthenelse{\value{kcounter}=20}{ト}{\ifthenelse{\value{kcounter}=21}{ナ}{\ifthenelse{\value{kcounter}=22}{ニ}{\ifthenelse{\value{kcounter}=23}{ヌ}{\ifthenelse{\value{kcounter}=24}{ネ}{\ifthenelse{\value{kcounter}=25}{ノ}{\ifthenelse{\value{kcounter}=26}{ハ}{\ifthenelse{\value{kcounter}=27}{ヒ}{\ifthenelse{\value{kcounter}=28}{フ}{\ifthenelse{\value{kcounter}=29}{ヘ}{\ifthenelse{\value{kcounter}=30}{ホ}{\ifthenelse{\value{kcounter}=31}{マ}{\ifthenelse{\value{kcounter}=32}{ミ}{\ifthenelse{\value{kcounter}=33}{ム}{\ifthenelse{\value{kcounter}=34}{メ}{\ifthenelse{\value{kcounter}=35}{モ}{\ifthenelse{\value{kcounter}=36}{ヤ}{\ifthenelse{\value{kcounter}=37}{ユ}{\ifthenelse{\value{kcounter}=38}{ヨ}{\ifthenelse{\value{kcounter}=39}{ラ}{\ifthenelse{\value{kcounter}=40}{リ}{\ifthenelse{\value{kcounter}=41}{ル}{\ifthenelse{\value{kcounter}=42}{レ}{\ifthenelse{\value{kcounter}=43}{ロ}{\ifthenelse{\value{kcounter}=44}{ワ}{・}}}}}}}}}}}}}}}}}}}}}}}}}}}}}}}}}}}}}}}}}}}}}}}}という。2点からの距離が等しい点は、垂直二等分線上にある。

1つの角を2等分する半直線を、その角の\framebox[1.5cm][c]{\raise 0.2ex\hbox{\textcircled{\scriptsize{\refstepcounter{kcounter}\ifthenelse{\value{kcounter}=1}{ア}{\ifthenelse{\value{kcounter}=2}{イ}{\ifthenelse{\value{kcounter}=3}{ウ}{\ifthenelse{\value{kcounter}=4}{エ}{\ifthenelse{\value{kcounter}=5}{オ} {\ifthenelse{\value{kcounter}=6}{カ}{\ifthenelse{\value{kcounter}=7}{キ}{\ifthenelse{\value{kcounter}=8}{ク}{\ifthenelse{\value{kcounter}=9}{ケ}{\ifthenelse{\value{kcounter}=10}{コ}{\ifthenelse{\value{kcounter}=11}{サ}{\ifthenelse{\value{kcounter}=12}{シ}{\ifthenelse{\value{kcounter}=13}{ス}{\ifthenelse{\value{kcounter}=14}{セ}{\ifthenelse{\value{kcounter}=15}{ソ}{\ifthenelse{\value{kcounter}=16}{タ}{\ifthenelse{\value{kcounter}=17}{チ}{\ifthenelse{\value{kcounter}=18}{ツ}{\ifthenelse{\value{kcounter}=19}{テ}{\ifthenelse{\value{kcounter}=20}{ト}{\ifthenelse{\value{kcounter}=21}{ナ}{\ifthenelse{\value{kcounter}=22}{ニ}{\ifthenelse{\value{kcounter}=23}{ヌ}{\ifthenelse{\value{kcounter}=24}{ネ}{\ifthenelse{\value{kcounter}=25}{ノ}{\ifthenelse{\value{kcounter}=26}{ハ}{\ifthenelse{\value{kcounter}=27}{ヒ}{\ifthenelse{\value{kcounter}=28}{フ}{\ifthenelse{\value{kcounter}=29}{ヘ}{\ifthenelse{\value{kcounter}=30}{ホ}{\ifthenelse{\value{kcounter}=31}{マ}{\ifthenelse{\value{kcounter}=32}{ミ}{\ifthenelse{\value{kcounter}=33}{ム}{\ifthenelse{\value{kcounter}=34}{メ}{\ifthenelse{\value{kcounter}=35}{モ}{\ifthenelse{\value{kcounter}=36}{ヤ}{\ifthenelse{\value{kcounter}=37}{ユ}{\ifthenelse{\value{kcounter}=38}{ヨ}{\ifthenelse{\value{kcounter}=39}{ラ}{\ifthenelse{\value{kcounter}=40}{リ}{\ifthenelse{\value{kcounter}=41}{ル}{\ifthenelse{\value{kcounter}=42}{レ}{\ifthenelse{\value{kcounter}=43}{ロ}{\ifthenelse{\value{kcounter}=44}{ワ}{・}}}}}}}}}}}}}}}}}}}}}}}}}}}}}}}}}}}}}}}}}}}}}}}}という。角の二等分線は、その角の\framebox[1.5cm][c]{\raise 0.2ex\hbox{\textcircled{\scriptsize{\refstepcounter{kcounter}\ifthenelse{\value{kcounter}=1}{ア}{\ifthenelse{\value{kcounter}=2}{イ}{\ifthenelse{\value{kcounter}=3}{ウ}{\ifthenelse{\value{kcounter}=4}{エ}{\ifthenelse{\value{kcounter}=5}{オ} {\ifthenelse{\value{kcounter}=6}{カ}{\ifthenelse{\value{kcounter}=7}{キ}{\ifthenelse{\value{kcounter}=8}{ク}{\ifthenelse{\value{kcounter}=9}{ケ}{\ifthenelse{\value{kcounter}=10}{コ}{\ifthenelse{\value{kcounter}=11}{サ}{\ifthenelse{\value{kcounter}=12}{シ}{\ifthenelse{\value{kcounter}=13}{ス}{\ifthenelse{\value{kcounter}=14}{セ}{\ifthenelse{\value{kcounter}=15}{ソ}{\ifthenelse{\value{kcounter}=16}{タ}{\ifthenelse{\value{kcounter}=17}{チ}{\ifthenelse{\value{kcounter}=18}{ツ}{\ifthenelse{\value{kcounter}=19}{テ}{\ifthenelse{\value{kcounter}=20}{ト}{\ifthenelse{\value{kcounter}=21}{ナ}{\ifthenelse{\value{kcounter}=22}{ニ}{\ifthenelse{\value{kcounter}=23}{ヌ}{\ifthenelse{\value{kcounter}=24}{ネ}{\ifthenelse{\value{kcounter}=25}{ノ}{\ifthenelse{\value{kcounter}=26}{ハ}{\ifthenelse{\value{kcounter}=27}{ヒ}{\ifthenelse{\value{kcounter}=28}{フ}{\ifthenelse{\value{kcounter}=29}{ヘ}{\ifthenelse{\value{kcounter}=30}{ホ}{\ifthenelse{\value{kcounter}=31}{マ}{\ifthenelse{\value{kcounter}=32}{ミ}{\ifthenelse{\value{kcounter}=33}{ム}{\ifthenelse{\value{kcounter}=34}{メ}{\ifthenelse{\value{kcounter}=35}{モ}{\ifthenelse{\value{kcounter}=36}{ヤ}{\ifthenelse{\value{kcounter}=37}{ユ}{\ifthenelse{\value{kcounter}=38}{ヨ}{\ifthenelse{\value{kcounter}=39}{ラ}{\ifthenelse{\value{kcounter}=40}{リ}{\ifthenelse{\value{kcounter}=41}{ル}{\ifthenelse{\value{kcounter}=42}{レ}{\ifthenelse{\value{kcounter}=43}{ロ}{\ifthenelse{\value{kcounter}=44}{ワ}{・}}}}}}}}}}}}}}}}}}}}}}}}}}}}}}}}}}}}}}}}}}}}}}}}である。角の二等分線上の点から角の二辺までの距離は等しい。また、角の内部にあって、その角の2辺までの距離が等しい点は、その角の二等分線上にある。

円周上の2点をA, Bとするとき、AからBまでの円周の部分を弧ABといい、\framebox[1.5cm][c]{\raise 0.2ex\hbox{\textcircled{\scriptsize{\refstepcounter{kcounter}\ifthenelse{\value{kcounter}=1}{ア}{\ifthenelse{\value{kcounter}=2}{イ}{\ifthenelse{\value{kcounter}=3}{ウ}{\ifthenelse{\value{kcounter}=4}{エ}{\ifthenelse{\value{kcounter}=5}{オ} {\ifthenelse{\value{kcounter}=6}{カ}{\ifthenelse{\value{kcounter}=7}{キ}{\ifthenelse{\value{kcounter}=8}{ク}{\ifthenelse{\value{kcounter}=9}{ケ}{\ifthenelse{\value{kcounter}=10}{コ}{\ifthenelse{\value{kcounter}=11}{サ}{\ifthenelse{\value{kcounter}=12}{シ}{\ifthenelse{\value{kcounter}=13}{ス}{\ifthenelse{\value{kcounter}=14}{セ}{\ifthenelse{\value{kcounter}=15}{ソ}{\ifthenelse{\value{kcounter}=16}{タ}{\ifthenelse{\value{kcounter}=17}{チ}{\ifthenelse{\value{kcounter}=18}{ツ}{\ifthenelse{\value{kcounter}=19}{テ}{\ifthenelse{\value{kcounter}=20}{ト}{\ifthenelse{\value{kcounter}=21}{ナ}{\ifthenelse{\value{kcounter}=22}{ニ}{\ifthenelse{\value{kcounter}=23}{ヌ}{\ifthenelse{\value{kcounter}=24}{ネ}{\ifthenelse{\value{kcounter}=25}{ノ}{\ifthenelse{\value{kcounter}=26}{ハ}{\ifthenelse{\value{kcounter}=27}{ヒ}{\ifthenelse{\value{kcounter}=28}{フ}{\ifthenelse{\value{kcounter}=29}{ヘ}{\ifthenelse{\value{kcounter}=30}{ホ}{\ifthenelse{\value{kcounter}=31}{マ}{\ifthenelse{\value{kcounter}=32}{ミ}{\ifthenelse{\value{kcounter}=33}{ム}{\ifthenelse{\value{kcounter}=34}{メ}{\ifthenelse{\value{kcounter}=35}{モ}{\ifthenelse{\value{kcounter}=36}{ヤ}{\ifthenelse{\value{kcounter}=37}{ユ}{\ifthenelse{\value{kcounter}=38}{ヨ}{\ifthenelse{\value{kcounter}=39}{ラ}{\ifthenelse{\value{kcounter}=40}{リ}{\ifthenelse{\value{kcounter}=41}{ル}{\ifthenelse{\value{kcounter}=42}{レ}{\ifthenelse{\value{kcounter}=43}{ロ}{\ifthenelse{\value{kcounter}=44}{ワ}{・}}}}}}}}}}}}}}}}}}}}}}}}}}}}}}}}}}}}}}}}}}}}}}}}と表す。円周上の2点を結ぶ線分を弦といい、両端がA, Bである弦を\framebox[1.5cm][c]{\raise 0.2ex\hbox{\textcircled{\scriptsize{\refstepcounter{kcounter}\ifthenelse{\value{kcounter}=1}{ア}{\ifthenelse{\value{kcounter}=2}{イ}{\ifthenelse{\value{kcounter}=3}{ウ}{\ifthenelse{\value{kcounter}=4}{エ}{\ifthenelse{\value{kcounter}=5}{オ} {\ifthenelse{\value{kcounter}=6}{カ}{\ifthenelse{\value{kcounter}=7}{キ}{\ifthenelse{\value{kcounter}=8}{ク}{\ifthenelse{\value{kcounter}=9}{ケ}{\ifthenelse{\value{kcounter}=10}{コ}{\ifthenelse{\value{kcounter}=11}{サ}{\ifthenelse{\value{kcounter}=12}{シ}{\ifthenelse{\value{kcounter}=13}{ス}{\ifthenelse{\value{kcounter}=14}{セ}{\ifthenelse{\value{kcounter}=15}{ソ}{\ifthenelse{\value{kcounter}=16}{タ}{\ifthenelse{\value{kcounter}=17}{チ}{\ifthenelse{\value{kcounter}=18}{ツ}{\ifthenelse{\value{kcounter}=19}{テ}{\ifthenelse{\value{kcounter}=20}{ト}{\ifthenelse{\value{kcounter}=21}{ナ}{\ifthenelse{\value{kcounter}=22}{ニ}{\ifthenelse{\value{kcounter}=23}{ヌ}{\ifthenelse{\value{kcounter}=24}{ネ}{\ifthenelse{\value{kcounter}=25}{ノ}{\ifthenelse{\value{kcounter}=26}{ハ}{\ifthenelse{\value{kcounter}=27}{ヒ}{\ifthenelse{\value{kcounter}=28}{フ}{\ifthenelse{\value{kcounter}=29}{ヘ}{\ifthenelse{\value{kcounter}=30}{ホ}{\ifthenelse{\value{kcounter}=31}{マ}{\ifthenelse{\value{kcounter}=32}{ミ}{\ifthenelse{\value{kcounter}=33}{ム}{\ifthenelse{\value{kcounter}=34}{メ}{\ifthenelse{\value{kcounter}=35}{モ}{\ifthenelse{\value{kcounter}=36}{ヤ}{\ifthenelse{\value{kcounter}=37}{ユ}{\ifthenelse{\value{kcounter}=38}{ヨ}{\ifthenelse{\value{kcounter}=39}{ラ}{\ifthenelse{\value{kcounter}=40}{リ}{\ifthenelse{\value{kcounter}=41}{ル}{\ifthenelse{\value{kcounter}=42}{レ}{\ifthenelse{\value{kcounter}=43}{ロ}{\ifthenelse{\value{kcounter}=44}{ワ}{・}}}}}}}}}}}}}}}}}}}}}}}}}}}}}}}}}}}}}}}}}}}}}}}}という。

円の中心を通る直線に垂直な直線を平行移動させていくと、1点だけで円と出あう場所がある。このとき、この直線は円に\framebox[1.5cm][c]{\raise 0.2ex\hbox{\textcircled{\scriptsize{\refstepcounter{kcounter}\ifthenelse{\value{kcounter}=1}{ア}{\ifthenelse{\value{kcounter}=2}{イ}{\ifthenelse{\value{kcounter}=3}{ウ}{\ifthenelse{\value{kcounter}=4}{エ}{\ifthenelse{\value{kcounter}=5}{オ} {\ifthenelse{\value{kcounter}=6}{カ}{\ifthenelse{\value{kcounter}=7}{キ}{\ifthenelse{\value{kcounter}=8}{ク}{\ifthenelse{\value{kcounter}=9}{ケ}{\ifthenelse{\value{kcounter}=10}{コ}{\ifthenelse{\value{kcounter}=11}{サ}{\ifthenelse{\value{kcounter}=12}{シ}{\ifthenelse{\value{kcounter}=13}{ス}{\ifthenelse{\value{kcounter}=14}{セ}{\ifthenelse{\value{kcounter}=15}{ソ}{\ifthenelse{\value{kcounter}=16}{タ}{\ifthenelse{\value{kcounter}=17}{チ}{\ifthenelse{\value{kcounter}=18}{ツ}{\ifthenelse{\value{kcounter}=19}{テ}{\ifthenelse{\value{kcounter}=20}{ト}{\ifthenelse{\value{kcounter}=21}{ナ}{\ifthenelse{\value{kcounter}=22}{ニ}{\ifthenelse{\value{kcounter}=23}{ヌ}{\ifthenelse{\value{kcounter}=24}{ネ}{\ifthenelse{\value{kcounter}=25}{ノ}{\ifthenelse{\value{kcounter}=26}{ハ}{\ifthenelse{\value{kcounter}=27}{ヒ}{\ifthenelse{\value{kcounter}=28}{フ}{\ifthenelse{\value{kcounter}=29}{ヘ}{\ifthenelse{\value{kcounter}=30}{ホ}{\ifthenelse{\value{kcounter}=31}{マ}{\ifthenelse{\value{kcounter}=32}{ミ}{\ifthenelse{\value{kcounter}=33}{ム}{\ifthenelse{\value{kcounter}=34}{メ}{\ifthenelse{\value{kcounter}=35}{モ}{\ifthenelse{\value{kcounter}=36}{ヤ}{\ifthenelse{\value{kcounter}=37}{ユ}{\ifthenelse{\value{kcounter}=38}{ヨ}{\ifthenelse{\value{kcounter}=39}{ラ}{\ifthenelse{\value{kcounter}=40}{リ}{\ifthenelse{\value{kcounter}=41}{ル}{\ifthenelse{\value{kcounter}=42}{レ}{\ifthenelse{\value{kcounter}=43}{ロ}{\ifthenelse{\value{kcounter}=44}{ワ}{・}}}}}}}}}}}}}}}}}}}}}}}}}}}}}}}}}}}}}}}}}}}}}}}}といい、この直線を円の\framebox[1.5cm][c]{\raise 0.2ex\hbox{\textcircled{\scriptsize{\refstepcounter{kcounter}\ifthenelse{\value{kcounter}=1}{ア}{\ifthenelse{\value{kcounter}=2}{イ}{\ifthenelse{\value{kcounter}=3}{ウ}{\ifthenelse{\value{kcounter}=4}{エ}{\ifthenelse{\value{kcounter}=5}{オ} {\ifthenelse{\value{kcounter}=6}{カ}{\ifthenelse{\value{kcounter}=7}{キ}{\ifthenelse{\value{kcounter}=8}{ク}{\ifthenelse{\value{kcounter}=9}{ケ}{\ifthenelse{\value{kcounter}=10}{コ}{\ifthenelse{\value{kcounter}=11}{サ}{\ifthenelse{\value{kcounter}=12}{シ}{\ifthenelse{\value{kcounter}=13}{ス}{\ifthenelse{\value{kcounter}=14}{セ}{\ifthenelse{\value{kcounter}=15}{ソ}{\ifthenelse{\value{kcounter}=16}{タ}{\ifthenelse{\value{kcounter}=17}{チ}{\ifthenelse{\value{kcounter}=18}{ツ}{\ifthenelse{\value{kcounter}=19}{テ}{\ifthenelse{\value{kcounter}=20}{ト}{\ifthenelse{\value{kcounter}=21}{ナ}{\ifthenelse{\value{kcounter}=22}{ニ}{\ifthenelse{\value{kcounter}=23}{ヌ}{\ifthenelse{\value{kcounter}=24}{ネ}{\ifthenelse{\value{kcounter}=25}{ノ}{\ifthenelse{\value{kcounter}=26}{ハ}{\ifthenelse{\value{kcounter}=27}{ヒ}{\ifthenelse{\value{kcounter}=28}{フ}{\ifthenelse{\value{kcounter}=29}{ヘ}{\ifthenelse{\value{kcounter}=30}{ホ}{\ifthenelse{\value{kcounter}=31}{マ}{\ifthenelse{\value{kcounter}=32}{ミ}{\ifthenelse{\value{kcounter}=33}{ム}{\ifthenelse{\value{kcounter}=34}{メ}{\ifthenelse{\value{kcounter}=35}{モ}{\ifthenelse{\value{kcounter}=36}{ヤ}{\ifthenelse{\value{kcounter}=37}{ユ}{\ifthenelse{\value{kcounter}=38}{ヨ}{\ifthenelse{\value{kcounter}=39}{ラ}{\ifthenelse{\value{kcounter}=40}{リ}{\ifthenelse{\value{kcounter}=41}{ル}{\ifthenelse{\value{kcounter}=42}{レ}{\ifthenelse{\value{kcounter}=43}{ロ}{\ifthenelse{\value{kcounter}=44}{ワ}{・}}}}}}}}}}}}}}}}}}}}}}}}}}}}}}}}}}}}}}}}}}}}}}}}、円と直線が接する点を\framebox[1.5cm][c]{\raise 0.2ex\hbox{\textcircled{\scriptsize{\refstepcounter{kcounter}\ifthenelse{\value{kcounter}=1}{ア}{\ifthenelse{\value{kcounter}=2}{イ}{\ifthenelse{\value{kcounter}=3}{ウ}{\ifthenelse{\value{kcounter}=4}{エ}{\ifthenelse{\value{kcounter}=5}{オ} {\ifthenelse{\value{kcounter}=6}{カ}{\ifthenelse{\value{kcounter}=7}{キ}{\ifthenelse{\value{kcounter}=8}{ク}{\ifthenelse{\value{kcounter}=9}{ケ}{\ifthenelse{\value{kcounter}=10}{コ}{\ifthenelse{\value{kcounter}=11}{サ}{\ifthenelse{\value{kcounter}=12}{シ}{\ifthenelse{\value{kcounter}=13}{ス}{\ifthenelse{\value{kcounter}=14}{セ}{\ifthenelse{\value{kcounter}=15}{ソ}{\ifthenelse{\value{kcounter}=16}{タ}{\ifthenelse{\value{kcounter}=17}{チ}{\ifthenelse{\value{kcounter}=18}{ツ}{\ifthenelse{\value{kcounter}=19}{テ}{\ifthenelse{\value{kcounter}=20}{ト}{\ifthenelse{\value{kcounter}=21}{ナ}{\ifthenelse{\value{kcounter}=22}{ニ}{\ifthenelse{\value{kcounter}=23}{ヌ}{\ifthenelse{\value{kcounter}=24}{ネ}{\ifthenelse{\value{kcounter}=25}{ノ}{\ifthenelse{\value{kcounter}=26}{ハ}{\ifthenelse{\value{kcounter}=27}{ヒ}{\ifthenelse{\value{kcounter}=28}{フ}{\ifthenelse{\value{kcounter}=29}{ヘ}{\ifthenelse{\value{kcounter}=30}{ホ}{\ifthenelse{\value{kcounter}=31}{マ}{\ifthenelse{\value{kcounter}=32}{ミ}{\ifthenelse{\value{kcounter}=33}{ム}{\ifthenelse{\value{kcounter}=34}{メ}{\ifthenelse{\value{kcounter}=35}{モ}{\ifthenelse{\value{kcounter}=36}{ヤ}{\ifthenelse{\value{kcounter}=37}{ユ}{\ifthenelse{\value{kcounter}=38}{ヨ}{\ifthenelse{\value{kcounter}=39}{ラ}{\ifthenelse{\value{kcounter}=40}{リ}{\ifthenelse{\value{kcounter}=41}{ル}{\ifthenelse{\value{kcounter}=42}{レ}{\ifthenelse{\value{kcounter}=43}{ロ}{\ifthenelse{\value{kcounter}=44}{ワ}{・}}}}}}}}}}}}}}}}}}}}}}}}}}}}}}}}}}}}}}}}}}}}}}}}という。円の接線は、接点を通る半径に\framebox[1.5cm][c]{\raise 0.2ex\hbox{\textcircled{\scriptsize{\refstepcounter{kcounter}\ifthenelse{\value{kcounter}=1}{ア}{\ifthenelse{\value{kcounter}=2}{イ}{\ifthenelse{\value{kcounter}=3}{ウ}{\ifthenelse{\value{kcounter}=4}{エ}{\ifthenelse{\value{kcounter}=5}{オ} {\ifthenelse{\value{kcounter}=6}{カ}{\ifthenelse{\value{kcounter}=7}{キ}{\ifthenelse{\value{kcounter}=8}{ク}{\ifthenelse{\value{kcounter}=9}{ケ}{\ifthenelse{\value{kcounter}=10}{コ}{\ifthenelse{\value{kcounter}=11}{サ}{\ifthenelse{\value{kcounter}=12}{シ}{\ifthenelse{\value{kcounter}=13}{ス}{\ifthenelse{\value{kcounter}=14}{セ}{\ifthenelse{\value{kcounter}=15}{ソ}{\ifthenelse{\value{kcounter}=16}{タ}{\ifthenelse{\value{kcounter}=17}{チ}{\ifthenelse{\value{kcounter}=18}{ツ}{\ifthenelse{\value{kcounter}=19}{テ}{\ifthenelse{\value{kcounter}=20}{ト}{\ifthenelse{\value{kcounter}=21}{ナ}{\ifthenelse{\value{kcounter}=22}{ニ}{\ifthenelse{\value{kcounter}=23}{ヌ}{\ifthenelse{\value{kcounter}=24}{ネ}{\ifthenelse{\value{kcounter}=25}{ノ}{\ifthenelse{\value{kcounter}=26}{ハ}{\ifthenelse{\value{kcounter}=27}{ヒ}{\ifthenelse{\value{kcounter}=28}{フ}{\ifthenelse{\value{kcounter}=29}{ヘ}{\ifthenelse{\value{kcounter}=30}{ホ}{\ifthenelse{\value{kcounter}=31}{マ}{\ifthenelse{\value{kcounter}=32}{ミ}{\ifthenelse{\value{kcounter}=33}{ム}{\ifthenelse{\value{kcounter}=34}{メ}{\ifthenelse{\value{kcounter}=35}{モ}{\ifthenelse{\value{kcounter}=36}{ヤ}{\ifthenelse{\value{kcounter}=37}{ユ}{\ifthenelse{\value{kcounter}=38}{ヨ}{\ifthenelse{\value{kcounter}=39}{ラ}{\ifthenelse{\value{kcounter}=40}{リ}{\ifthenelse{\value{kcounter}=41}{ル}{\ifthenelse{\value{kcounter}=42}{レ}{\ifthenelse{\value{kcounter}=43}{ロ}{\ifthenelse{\value{kcounter}=44}{ワ}{・}}}}}}}}}}}}}}}}}}}}}}}}}}}}}}}}}}}}}}}}}}}}}}}}である。

弧の両端を通る2つの半径とその弧で囲まれた図形を\framebox[1.5cm][c]{\raise 0.2ex\hbox{\textcircled{\scriptsize{\refstepcounter{kcounter}\ifthenelse{\value{kcounter}=1}{ア}{\ifthenelse{\value{kcounter}=2}{イ}{\ifthenelse{\value{kcounter}=3}{ウ}{\ifthenelse{\value{kcounter}=4}{エ}{\ifthenelse{\value{kcounter}=5}{オ} {\ifthenelse{\value{kcounter}=6}{カ}{\ifthenelse{\value{kcounter}=7}{キ}{\ifthenelse{\value{kcounter}=8}{ク}{\ifthenelse{\value{kcounter}=9}{ケ}{\ifthenelse{\value{kcounter}=10}{コ}{\ifthenelse{\value{kcounter}=11}{サ}{\ifthenelse{\value{kcounter}=12}{シ}{\ifthenelse{\value{kcounter}=13}{ス}{\ifthenelse{\value{kcounter}=14}{セ}{\ifthenelse{\value{kcounter}=15}{ソ}{\ifthenelse{\value{kcounter}=16}{タ}{\ifthenelse{\value{kcounter}=17}{チ}{\ifthenelse{\value{kcounter}=18}{ツ}{\ifthenelse{\value{kcounter}=19}{テ}{\ifthenelse{\value{kcounter}=20}{ト}{\ifthenelse{\value{kcounter}=21}{ナ}{\ifthenelse{\value{kcounter}=22}{ニ}{\ifthenelse{\value{kcounter}=23}{ヌ}{\ifthenelse{\value{kcounter}=24}{ネ}{\ifthenelse{\value{kcounter}=25}{ノ}{\ifthenelse{\value{kcounter}=26}{ハ}{\ifthenelse{\value{kcounter}=27}{ヒ}{\ifthenelse{\value{kcounter}=28}{フ}{\ifthenelse{\value{kcounter}=29}{ヘ}{\ifthenelse{\value{kcounter}=30}{ホ}{\ifthenelse{\value{kcounter}=31}{マ}{\ifthenelse{\value{kcounter}=32}{ミ}{\ifthenelse{\value{kcounter}=33}{ム}{\ifthenelse{\value{kcounter}=34}{メ}{\ifthenelse{\value{kcounter}=35}{モ}{\ifthenelse{\value{kcounter}=36}{ヤ}{\ifthenelse{\value{kcounter}=37}{ユ}{\ifthenelse{\value{kcounter}=38}{ヨ}{\ifthenelse{\value{kcounter}=39}{ラ}{\ifthenelse{\value{kcounter}=40}{リ}{\ifthenelse{\value{kcounter}=41}{ル}{\ifthenelse{\value{kcounter}=42}{レ}{\ifthenelse{\value{kcounter}=43}{ロ}{\ifthenelse{\value{kcounter}=44}{ワ}{・}}}}}}}}}}}}}}}}}}}}}}}}}}}}}}}}}}}}}}}}}}}}}}}}という。おうぎ形で、2つの半径のつくる角を\framebox[1.5cm][c]{\raise 0.2ex\hbox{\textcircled{\scriptsize{\refstepcounter{kcounter}\ifthenelse{\value{kcounter}=1}{ア}{\ifthenelse{\value{kcounter}=2}{イ}{\ifthenelse{\value{kcounter}=3}{ウ}{\ifthenelse{\value{kcounter}=4}{エ}{\ifthenelse{\value{kcounter}=5}{オ} {\ifthenelse{\value{kcounter}=6}{カ}{\ifthenelse{\value{kcounter}=7}{キ}{\ifthenelse{\value{kcounter}=8}{ク}{\ifthenelse{\value{kcounter}=9}{ケ}{\ifthenelse{\value{kcounter}=10}{コ}{\ifthenelse{\value{kcounter}=11}{サ}{\ifthenelse{\value{kcounter}=12}{シ}{\ifthenelse{\value{kcounter}=13}{ス}{\ifthenelse{\value{kcounter}=14}{セ}{\ifthenelse{\value{kcounter}=15}{ソ}{\ifthenelse{\value{kcounter}=16}{タ}{\ifthenelse{\value{kcounter}=17}{チ}{\ifthenelse{\value{kcounter}=18}{ツ}{\ifthenelse{\value{kcounter}=19}{テ}{\ifthenelse{\value{kcounter}=20}{ト}{\ifthenelse{\value{kcounter}=21}{ナ}{\ifthenelse{\value{kcounter}=22}{ニ}{\ifthenelse{\value{kcounter}=23}{ヌ}{\ifthenelse{\value{kcounter}=24}{ネ}{\ifthenelse{\value{kcounter}=25}{ノ}{\ifthenelse{\value{kcounter}=26}{ハ}{\ifthenelse{\value{kcounter}=27}{ヒ}{\ifthenelse{\value{kcounter}=28}{フ}{\ifthenelse{\value{kcounter}=29}{ヘ}{\ifthenelse{\value{kcounter}=30}{ホ}{\ifthenelse{\value{kcounter}=31}{マ}{\ifthenelse{\value{kcounter}=32}{ミ}{\ifthenelse{\value{kcounter}=33}{ム}{\ifthenelse{\value{kcounter}=34}{メ}{\ifthenelse{\value{kcounter}=35}{モ}{\ifthenelse{\value{kcounter}=36}{ヤ}{\ifthenelse{\value{kcounter}=37}{ユ}{\ifthenelse{\value{kcounter}=38}{ヨ}{\ifthenelse{\value{kcounter}=39}{ラ}{\ifthenelse{\value{kcounter}=40}{リ}{\ifthenelse{\value{kcounter}=41}{ル}{\ifthenelse{\value{kcounter}=42}{レ}{\ifthenelse{\value{kcounter}=43}{ロ}{\ifthenelse{\value{kcounter}=44}{ワ}{・}}}}}}}}}}}}}}}}}}}}}}}}}}}}}}}}}}}}}}}}}}}}}}}}という。

\begin{center}
\def\@captype{figure}
\includegraphics[height=52mm]{media_ku/image4.png}

\end{center}

上の$\raise 0.2ex\hbox{\textcircled{\scriptsize{ア}}} \sim \raise 0.2ex\hbox{\textcircled{\scriptsize{エ}}}$の立体のように、平面だけで囲まれた立体を\framebox[1.5cm][c]{\raise 0.2ex\hbox{\textcircled{\scriptsize{\refstepcounter{kcounter}\ifthenelse{\value{kcounter}=1}{ア}{\ifthenelse{\value{kcounter}=2}{イ}{\ifthenelse{\value{kcounter}=3}{ウ}{\ifthenelse{\value{kcounter}=4}{エ}{\ifthenelse{\value{kcounter}=5}{オ} {\ifthenelse{\value{kcounter}=6}{カ}{\ifthenelse{\value{kcounter}=7}{キ}{\ifthenelse{\value{kcounter}=8}{ク}{\ifthenelse{\value{kcounter}=9}{ケ}{\ifthenelse{\value{kcounter}=10}{コ}{\ifthenelse{\value{kcounter}=11}{サ}{\ifthenelse{\value{kcounter}=12}{シ}{\ifthenelse{\value{kcounter}=13}{ス}{\ifthenelse{\value{kcounter}=14}{セ}{\ifthenelse{\value{kcounter}=15}{ソ}{\ifthenelse{\value{kcounter}=16}{タ}{\ifthenelse{\value{kcounter}=17}{チ}{\ifthenelse{\value{kcounter}=18}{ツ}{\ifthenelse{\value{kcounter}=19}{テ}{\ifthenelse{\value{kcounter}=20}{ト}{\ifthenelse{\value{kcounter}=21}{ナ}{\ifthenelse{\value{kcounter}=22}{ニ}{\ifthenelse{\value{kcounter}=23}{ヌ}{\ifthenelse{\value{kcounter}=24}{ネ}{\ifthenelse{\value{kcounter}=25}{ノ}{\ifthenelse{\value{kcounter}=26}{ハ}{\ifthenelse{\value{kcounter}=27}{ヒ}{\ifthenelse{\value{kcounter}=28}{フ}{\ifthenelse{\value{kcounter}=29}{ヘ}{\ifthenelse{\value{kcounter}=30}{ホ}{\ifthenelse{\value{kcounter}=31}{マ}{\ifthenelse{\value{kcounter}=32}{ミ}{\ifthenelse{\value{kcounter}=33}{ム}{\ifthenelse{\value{kcounter}=34}{メ}{\ifthenelse{\value{kcounter}=35}{モ}{\ifthenelse{\value{kcounter}=36}{ヤ}{\ifthenelse{\value{kcounter}=37}{ユ}{\ifthenelse{\value{kcounter}=38}{ヨ}{\ifthenelse{\value{kcounter}=39}{ラ}{\ifthenelse{\value{kcounter}=40}{リ}{\ifthenelse{\value{kcounter}=41}{ル}{\ifthenelse{\value{kcounter}=42}{レ}{\ifthenelse{\value{kcounter}=43}{ロ}{\ifthenelse{\value{kcounter}=44}{ワ}{・}}}}}}}}}}}}}}}}}}}}}}}}}}}}}}}}}}}}}}}}}}}}}}}}という。また、$\raise 0.2ex\hbox{\textcircled{\scriptsize{イ}}}$や$\raise 0.2ex\hbox{\textcircled{\scriptsize{エ}}}$のような立体を\framebox[1.5cm][c]{\raise 0.2ex\hbox{\textcircled{\scriptsize{\refstepcounter{kcounter}\ifthenelse{\value{kcounter}=1}{ア}{\ifthenelse{\value{kcounter}=2}{イ}{\ifthenelse{\value{kcounter}=3}{ウ}{\ifthenelse{\value{kcounter}=4}{エ}{\ifthenelse{\value{kcounter}=5}{オ} {\ifthenelse{\value{kcounter}=6}{カ}{\ifthenelse{\value{kcounter}=7}{キ}{\ifthenelse{\value{kcounter}=8}{ク}{\ifthenelse{\value{kcounter}=9}{ケ}{\ifthenelse{\value{kcounter}=10}{コ}{\ifthenelse{\value{kcounter}=11}{サ}{\ifthenelse{\value{kcounter}=12}{シ}{\ifthenelse{\value{kcounter}=13}{ス}{\ifthenelse{\value{kcounter}=14}{セ}{\ifthenelse{\value{kcounter}=15}{ソ}{\ifthenelse{\value{kcounter}=16}{タ}{\ifthenelse{\value{kcounter}=17}{チ}{\ifthenelse{\value{kcounter}=18}{ツ}{\ifthenelse{\value{kcounter}=19}{テ}{\ifthenelse{\value{kcounter}=20}{ト}{\ifthenelse{\value{kcounter}=21}{ナ}{\ifthenelse{\value{kcounter}=22}{ニ}{\ifthenelse{\value{kcounter}=23}{ヌ}{\ifthenelse{\value{kcounter}=24}{ネ}{\ifthenelse{\value{kcounter}=25}{ノ}{\ifthenelse{\value{kcounter}=26}{ハ}{\ifthenelse{\value{kcounter}=27}{ヒ}{\ifthenelse{\value{kcounter}=28}{フ}{\ifthenelse{\value{kcounter}=29}{ヘ}{\ifthenelse{\value{kcounter}=30}{ホ}{\ifthenelse{\value{kcounter}=31}{マ}{\ifthenelse{\value{kcounter}=32}{ミ}{\ifthenelse{\value{kcounter}=33}{ム}{\ifthenelse{\value{kcounter}=34}{メ}{\ifthenelse{\value{kcounter}=35}{モ}{\ifthenelse{\value{kcounter}=36}{ヤ}{\ifthenelse{\value{kcounter}=37}{ユ}{\ifthenelse{\value{kcounter}=38}{ヨ}{\ifthenelse{\value{kcounter}=39}{ラ}{\ifthenelse{\value{kcounter}=40}{リ}{\ifthenelse{\value{kcounter}=41}{ル}{\ifthenelse{\value{kcounter}=42}{レ}{\ifthenelse{\value{kcounter}=43}{ロ}{\ifthenelse{\value{kcounter}=44}{ワ}{・}}}}}}}}}}}}}}}}}}}}}}}}}}}}}}}}}}}}}}}}}}}}}}}}という。$\raise 0.2ex\hbox{\textcircled{\scriptsize{カ}}}$のような立体を\framebox[1.5cm][c]{\raise 0.2ex\hbox{\textcircled{\scriptsize{\refstepcounter{kcounter}\ifthenelse{\value{kcounter}=1}{ア}{\ifthenelse{\value{kcounter}=2}{イ}{\ifthenelse{\value{kcounter}=3}{ウ}{\ifthenelse{\value{kcounter}=4}{エ}{\ifthenelse{\value{kcounter}=5}{オ} {\ifthenelse{\value{kcounter}=6}{カ}{\ifthenelse{\value{kcounter}=7}{キ}{\ifthenelse{\value{kcounter}=8}{ク}{\ifthenelse{\value{kcounter}=9}{ケ}{\ifthenelse{\value{kcounter}=10}{コ}{\ifthenelse{\value{kcounter}=11}{サ}{\ifthenelse{\value{kcounter}=12}{シ}{\ifthenelse{\value{kcounter}=13}{ス}{\ifthenelse{\value{kcounter}=14}{セ}{\ifthenelse{\value{kcounter}=15}{ソ}{\ifthenelse{\value{kcounter}=16}{タ}{\ifthenelse{\value{kcounter}=17}{チ}{\ifthenelse{\value{kcounter}=18}{ツ}{\ifthenelse{\value{kcounter}=19}{テ}{\ifthenelse{\value{kcounter}=20}{ト}{\ifthenelse{\value{kcounter}=21}{ナ}{\ifthenelse{\value{kcounter}=22}{ニ}{\ifthenelse{\value{kcounter}=23}{ヌ}{\ifthenelse{\value{kcounter}=24}{ネ}{\ifthenelse{\value{kcounter}=25}{ノ}{\ifthenelse{\value{kcounter}=26}{ハ}{\ifthenelse{\value{kcounter}=27}{ヒ}{\ifthenelse{\value{kcounter}=28}{フ}{\ifthenelse{\value{kcounter}=29}{ヘ}{\ifthenelse{\value{kcounter}=30}{ホ}{\ifthenelse{\value{kcounter}=31}{マ}{\ifthenelse{\value{kcounter}=32}{ミ}{\ifthenelse{\value{kcounter}=33}{ム}{\ifthenelse{\value{kcounter}=34}{メ}{\ifthenelse{\value{kcounter}=35}{モ}{\ifthenelse{\value{kcounter}=36}{ヤ}{\ifthenelse{\value{kcounter}=37}{ユ}{\ifthenelse{\value{kcounter}=38}{ヨ}{\ifthenelse{\value{kcounter}=39}{ラ}{\ifthenelse{\value{kcounter}=40}{リ}{\ifthenelse{\value{kcounter}=41}{ル}{\ifthenelse{\value{kcounter}=42}{レ}{\ifthenelse{\value{kcounter}=43}{ロ}{\ifthenelse{\value{kcounter}=44}{ワ}{・}}}}}}}}}}}}}}}}}}}}}}}}}}}}}}}}}}}}}}}}}}}}}}}}という。

底面が正三角形、正方形、$\cdots$で、側面がすべて合同な長方形である角柱をそれぞれ、正三角柱、正四角柱、$\cdots$という。また、底面が正三角形、正方形、$\cdots$で、側面がすべて合同な二等辺三角形である角錐を、それぞれ正三角錐、正四角錐、$\cdots$という。

立方体のように、多面体で次の2つの性質をもち、へこみのないものを\framebox[1.5cm][c]{\raise 0.2ex\hbox{\textcircled{\scriptsize{\refstepcounter{kcounter}\ifthenelse{\value{kcounter}=1}{ア}{\ifthenelse{\value{kcounter}=2}{イ}{\ifthenelse{\value{kcounter}=3}{ウ}{\ifthenelse{\value{kcounter}=4}{エ}{\ifthenelse{\value{kcounter}=5}{オ} {\ifthenelse{\value{kcounter}=6}{カ}{\ifthenelse{\value{kcounter}=7}{キ}{\ifthenelse{\value{kcounter}=8}{ク}{\ifthenelse{\value{kcounter}=9}{ケ}{\ifthenelse{\value{kcounter}=10}{コ}{\ifthenelse{\value{kcounter}=11}{サ}{\ifthenelse{\value{kcounter}=12}{シ}{\ifthenelse{\value{kcounter}=13}{ス}{\ifthenelse{\value{kcounter}=14}{セ}{\ifthenelse{\value{kcounter}=15}{ソ}{\ifthenelse{\value{kcounter}=16}{タ}{\ifthenelse{\value{kcounter}=17}{チ}{\ifthenelse{\value{kcounter}=18}{ツ}{\ifthenelse{\value{kcounter}=19}{テ}{\ifthenelse{\value{kcounter}=20}{ト}{\ifthenelse{\value{kcounter}=21}{ナ}{\ifthenelse{\value{kcounter}=22}{ニ}{\ifthenelse{\value{kcounter}=23}{ヌ}{\ifthenelse{\value{kcounter}=24}{ネ}{\ifthenelse{\value{kcounter}=25}{ノ}{\ifthenelse{\value{kcounter}=26}{ハ}{\ifthenelse{\value{kcounter}=27}{ヒ}{\ifthenelse{\value{kcounter}=28}{フ}{\ifthenelse{\value{kcounter}=29}{ヘ}{\ifthenelse{\value{kcounter}=30}{ホ}{\ifthenelse{\value{kcounter}=31}{マ}{\ifthenelse{\value{kcounter}=32}{ミ}{\ifthenelse{\value{kcounter}=33}{ム}{\ifthenelse{\value{kcounter}=34}{メ}{\ifthenelse{\value{kcounter}=35}{モ}{\ifthenelse{\value{kcounter}=36}{ヤ}{\ifthenelse{\value{kcounter}=37}{ユ}{\ifthenelse{\value{kcounter}=38}{ヨ}{\ifthenelse{\value{kcounter}=39}{ラ}{\ifthenelse{\value{kcounter}=40}{リ}{\ifthenelse{\value{kcounter}=41}{ル}{\ifthenelse{\value{kcounter}=42}{レ}{\ifthenelse{\value{kcounter}=43}{ロ}{\ifthenelse{\value{kcounter}=44}{ワ}{・}}}}}}}}}}}}}}}}}}}}}}}}}}}}}}}}}}}}}}}}}}}}}}}}という。

\begin{itemize}
\item どの面もすべて合同な正多角形である。
\item どの頂点にも面が同じ数だけ集まっている。
\end{itemize}

\vfill

\noindent\fbox{\large\makebox[1em]{\text{\refstepcounter{kaunta}%
\arabic{kaunta}}}} \hspace{1pt}下の図の立体について、次の問に答えなさい。

%
\begin{flushright}%
\footnotesize{<知・技$2 \times 6$点>}%
\end{flushright}%


\begin{center}
\def\@captype{figure}
\includegraphics[height=40mm]{media/tu2.jpg}

\end{center}

(\text{\refstepcounter{skaunta}%
\arabic{skaunta}})\hspace{2.5pt}$\raise 0.2ex\hbox{\textcircled{\scriptsize{ア}}}, \, \raise 0.2ex\hbox{\textcircled{\scriptsize{ウ}}}, \, \raise 0.2ex\hbox{\textcircled{\scriptsize{エ}}}$の立体の名前と、それが何面体であるかを答えなさい。

(\text{\refstepcounter{skaunta}%
\arabic{skaunta}})\hspace{2.5pt}$\raise 0.2ex\hbox{\textcircled{\scriptsize{オ}}}, \, \raise 0.2ex\hbox{\textcircled{\scriptsize{カ}}}$の立体の名前と、側面の形を答えなさい。

(\text{\refstepcounter{skaunta}%
\arabic{skaunta}})\hspace{2.5pt}$\raise 0.2ex\hbox{\textcircled{\scriptsize{ク}}}$の立体の名前を答えなさい。

\setcounter{skaunta}{0}

\vfill

\noindent\fbox{\large\makebox[1em]{\text{\refstepcounter{kaunta}%
\arabic{kaunta}}}} \hspace{1pt}正四面体、正六面体、正八面体について、下の表の空欄をうめて、表を完成させなさい。

%
\begin{flushright}%
\footnotesize{<知・技$2 \times 4$点>}%
\end{flushright}%


\begin{center}
\def\@captype{figure}
\includegraphics[height=40mm]{media/tu3.jpg}


\renewcommand{\arraystretch}{1.5}
\begin{tabular}{|c|c|c|c|}
\hline
 & 面の数 & 辺の数 & 頂点の数 \\
\hline
正四面体 & (\text{\refstepcounter{skaunta}%
\arabic{skaunta}})\hspace{2.5pt}& 6 & 4\\
\hline
正六面体 & 6 & 12 & (\text{\refstepcounter{skaunta}%
\arabic{skaunta}})\hspace{2.5pt}\\
\hline
正八面体 & (\text{\refstepcounter{skaunta}%
\arabic{skaunta}})\hspace{2.5pt}& (\text{\refstepcounter{skaunta}%
\arabic{skaunta}})\hspace{2.5pt}& 6\\
\hline
\end{tabular}

\end{center}

\vfill

\newpage

\begin{multicols}{2}

\noindent\fbox{\large\makebox[1em]{\text{\refstepcounter{kaunta}%
\arabic{kaunta}}}} \hspace{1pt}右の長方形ABCDについて、線分ADと線分BCの関係を、記号を使って2通りの方法で表しなさい。

%
\begin{flushright}%
\footnotesize{<知・技$4$点>}%
\end{flushright}%


\columnbreak

\begin{center}
\def\@captype{figure}
\includegraphics{media/image1.png}

\end{center}

\end{multicols}

\vspace*{\fill}

\begin{multicols}{2}

\noindent\fbox{\large\makebox[1em]{\text{\refstepcounter{kaunta}%
\arabic{kaunta}}}} \hspace{1pt}右の図形について、次の問に答えなさい。

%
\begin{flushright}%
\footnotesize{<知・技$2 \times 2$点>}%
\end{flushright}%


\setcounter{skaunta}{0}

(\text{\refstepcounter{skaunta}%
\arabic{skaunta}})\hspace{2.5pt}三角形を直線$l$を対称の軸として対称移動させた図形をかきなさい。

\vspace*{\fill}

\columnbreak

\begin{center}
\def\@captype{figure}
\includegraphics[width=30mm]{media/tu4.png}


\end{center}

\end{multicols}

\begin{multicols}{2}

(\text{\refstepcounter{skaunta}%
\arabic{skaunta}})\hspace{2.5pt}三角形を点Oを中心として$\ang{180}$回転移動させた図形をかきなさい。

\vspace*{\fill}

\columnbreak

\begin{center}
\def\@captype{figure}
\includegraphics[width=30mm]{media/image3.png}

\end{center}

\end{multicols}

\begin{multicols}{2}

\vspace*{\fill}

\setcounter{skaunta}{0}
\noindent\fbox{\large\makebox[1em]{\text{\refstepcounter{kaunta}%
\arabic{kaunta}}}} \hspace{1pt}右の図の多角形ABCDEFは、1辺が$2\si{cm}$の正六角形で、点Oは3つの対角線の交点です。

%
\begin{flushright}%
\footnotesize{<(1)(2)知・技$2 \times 2$点、(3)思・判・表2点>}%
\end{flushright}%


(\text{\refstepcounter{skaunta}%
\arabic{skaunta}})\hspace{2.5pt}$\triangle$EFOは$\triangle$BCOを、対角線ADを対称の軸として対称移動させた図形とみることができます。このとき、ADとBFの関係を式で表しなさい。

\columnbreak

\vspace*{\fill}

\begin{center}
\def\@captype{figure}
\includegraphics[width=30mm]{media/image5.png}

\end{center}

\vspace*{\fill}

\end{multicols}

\vspace*{\fill}

(\text{\refstepcounter{skaunta}%
\arabic{skaunta}})\hspace{2.5pt}$\triangle$BCOは、平行移動だけで$\triangle$AOFに重ねることができます。このときの移動の距離は何$\si{cm}$ですか。

\vspace*{\fill}

(\text{\refstepcounter{skaunta}%
\arabic{skaunta}})\hspace{2.5pt}$\triangle$BCOは、回転移動だけで$\triangle$AOFに重ねることができます。どのように移動したらよいか、その方法を説明しなさい。

\vspace*{\fill}

\setcounter{skaunta}{0}

\newpage

\noindent\fbox{\large\makebox[1em]{\text{\refstepcounter{kaunta}%
\arabic{kaunta}}}} \hspace{1pt}次の作図をしなさい。ただし、作図に用いた線は残しておくこと。

%
\begin{flushright}%
\footnotesize{<知・技$2 \times 5$点>}%
\end{flushright}%


(\text{\refstepcounter{skaunta}%
\arabic{skaunta}})\hspace{2.5pt}$\triangle$ABCで、辺BCを底辺としたときの高さAP

\begin{center}
\def\@captype{figure}
\includegraphics[width=50mm]{media/image10.png}

\end{center}

(\text{\refstepcounter{skaunta}%
\arabic{skaunta}})\hspace{2.5pt}線分ABの中点M

\vspace{30mm}

\begin{center}
\def\@captype{figure}
\includegraphics[width=50mm]{media/image11.png}

\end{center}

(\text{\refstepcounter{skaunta}%
\arabic{skaunta}})\hspace{2.5pt}$\angle$AOBの二等分線OC

\begin{center}
\def\@captype{figure}
\includegraphics[width=50mm]{media/image8.png}

\end{center}

(\text{\refstepcounter{skaunta}%
\arabic{skaunta}})\hspace{2.5pt}線分ABの垂直二等分線

\vspace{30mm}

\begin{center}
\def\@captype{figure}
\includegraphics[width=50mm]{media/image11.png}

\end{center}

(\text{\refstepcounter{skaunta}%
\arabic{skaunta}})\hspace{2.5pt}辺AC, AB, BDに接する円O

\vspace{15mm}

\begin{center}
\def\@captype{figure}
\includegraphics[width=50mm]{media/image191.png}

\end{center}

\newpage

\begin{multicols}{2}

\setcounter{skaunta}{0}
\noindent\fbox{\large\makebox[1em]{\text{\refstepcounter{kaunta}%
\arabic{kaunta}}}} \hspace{1pt}さやかさんは、右の四角形\raise 0.2ex\hbox{\textcircled{\scriptsize{1}}}を、ある点Oを回転の中心として、回転移動させた図形を並べて、模様をつくることにしました。右の四角形\raise 0.2ex\hbox{\textcircled{\scriptsize{2}}}は、四角形\raise 0.2ex\hbox{\textcircled{\scriptsize{1}}}を、点Oを回転の中心として時計回りに$\ang{120}$移動させた図形です。

%
\begin{flushright}%
\footnotesize{<(1)知・技2点、(2)思・判・表4点>}%
\end{flushright}%


\columnbreak

\begin{center}
\def\@captype{figure}
\includegraphics{media/image12.png}

\end{center}

\end{multicols}

(\text{\refstepcounter{skaunta}%
\arabic{skaunta}})\hspace{2.5pt}四角形\raise 0.2ex\hbox{\textcircled{\scriptsize{2}}}の印を付けた角(\def\@captype{figure}\includegraphics[scale=1.0]{media/image13.png})に対応する角を、次のア$\sim$エの中から選び、記号で答えなさい。

\begin{itemize}
\item[ア] $\angle$A
\item[イ] $\angle$B
\item[ウ] $\angle$C
\item[エ] $\angle$D
\end{itemize}



(\text{\refstepcounter{skaunta}%
\arabic{skaunta}})\hspace{2.5pt}さやかさんは、四角形\raise 0.2ex\hbox{\textcircled{\scriptsize{1}}}を四角形$\raise 0.2ex\hbox{\textcircled{\scriptsize{2}}}$に回転移動させた後、回転の中心Oの位置が分からなくなりました。回転の中心Oの求め方を、「垂直二等分線」の言葉を使って説明しなさい。

\vspace{20mm}

\noindent\fbox{\large\makebox[1em]{\text{\refstepcounter{kaunta}%
\arabic{kaunta}}}} \hspace{1pt}$\ang{75}$を作図しなさい。

%
\begin{flushright}%
\footnotesize{<思・判・表5点>}%
\end{flushright}%


\newpage

\setcounter{skaunta}{0}
\noindent\fbox{\large\makebox[1em]{\text{\refstepcounter{kaunta}%
\arabic{kaunta}}}} \hspace{1pt}次のおうぎ形に関する問に答えなさい。

%
\begin{flushright}%
\footnotesize{<知・技$2 \times 7$点>}%
\end{flushright}%


(\text{\refstepcounter{skaunta}%
\arabic{skaunta}})\hspace{2.5pt}半径が$6\si{cm},$中心角が$\ang{30}$のおうぎ形の面積

\vspace{20mm}

(\text{\refstepcounter{skaunta}%
\arabic{skaunta}})\hspace{2.5pt}半径が$10\si{cm},$中心角が$\ang{108}$のおうぎ形の弧の長さ

\vspace{20mm}

(\text{\refstepcounter{skaunta}%
\arabic{skaunta}})\hspace{2.5pt}弧の長さが$10 \pi,$中心角が$\ang{120}$のおうぎ形の面積

\vspace{20mm}

(\text{\refstepcounter{skaunta}%
\arabic{skaunta}})\hspace{2.5pt}半径が$12\si{cm},$面積が$72 \pi \si{cm}^2$のおうぎ形の中心角

\vspace{20mm}

(\text{\refstepcounter{skaunta}%
\arabic{skaunta}})\hspace{2.5pt}弧の長さが$6 \pi \si{cm}, $中心角が$\ang{120}$のおうぎ形の半径

\vspace{20mm}

(\text{\refstepcounter{skaunta}%
\arabic{skaunta}})\hspace{2.5pt}中心角が$\ang{90} \si{cm},$弧の長さが$9 \pi \si{cm}$のおうぎ形の面積

\vspace{20mm}

\begin{multicols}{2}

(\text{\refstepcounter{skaunta}%
\arabic{skaunta}})\hspace{2.5pt}右の図は、2つのおうぎ形を組み合わせたものである。色をつけた部分について、次のものを求めなさい。

\columnbreak

\begin{center}
\def\@captype{figure}
\includegraphics[height=30mm]{media/tu1.jpg}

\end{center}

\end{multicols}

\vfill

\newpage






















\end{flushleft}

\end{document}
