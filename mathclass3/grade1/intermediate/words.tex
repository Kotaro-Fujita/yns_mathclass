% Options for packages loaded elsewhere
\PassOptionsToPackage{unicode}{hyperref}
\PassOptionsToPackage{hyphens}{url}
%
\documentclass[
  14pt,a4paper,lualatex,ja=standard]{bxjsarticle}
\usepackage{lmodern}
\usepackage{amsmath}
\usepackage{ifxetex,ifluatex}
\ifnum 0\ifxetex 1\fi\ifluatex 1\fi=0 % if pdftex
  \usepackage[T1]{fontenc}
  \usepackage[utf8]{inputenc}
  \usepackage{textcomp} % provide euro and other symbols
  \usepackage{amssymb}
\else % if luatex or xetex
  \usepackage{unicode-math}
  \defaultfontfeatures{Scale=MatchLowercase}
  \defaultfontfeatures[\rmfamily]{Ligatures=TeX,Scale=1}
\fi
% Use upquote if available, for straight quotes in verbatim environments
\IfFileExists{upquote.sty}{\usepackage{upquote}}{}
\IfFileExists{microtype.sty}{% use microtype if available
  \usepackage[]{microtype}
  \UseMicrotypeSet[protrusion]{basicmath} % disable protrusion for tt fonts
}{}
\makeatletter
\@ifundefined{KOMAClassName}{% if non-KOMA class
  \IfFileExists{parskip.sty}{%
    \usepackage{parskip}
  }{% else
    \setlength{\parindent}{0pt}
    \setlength{\parskip}{6pt plus 2pt minus 1pt}}
}{% if KOMA class
  \KOMAoptions{parskip=half}}
\makeatother
\usepackage{xcolor}
\IfFileExists{xurl.sty}{\usepackage{xurl}}{} % add URL line breaks if available
\IfFileExists{bookmark.sty}{\usepackage{bookmark}}{\usepackage{hyperref}}
\hypersetup{
  hidelinks,
  pdfcreator={LaTeX via pandoc}}
\urlstyle{same} % disable monospaced font for URLs
\usepackage{graphicx}
\makeatletter
\def\maxwidth{\ifdim\Gin@nat@width>\linewidth\linewidth\else\Gin@nat@width\fi}
\def\maxheight{\ifdim\Gin@nat@height>\textheight\textheight\else\Gin@nat@height\fi}
\makeatother
% Scale images if necessary, so that they will not overflow the page
% margins by default, and it is still possible to overwrite the defaults
% using explicit options in \includegraphics[width, height, ...]{}
\setkeys{Gin}{width=\maxwidth,height=\maxheight,keepaspectratio}
% Set default figure placement to htbp
\makeatletter
\def\fps@figure{htbp}
\makeatother
\setlength{\emergencystretch}{3em} % prevent overfull lines
\providecommand{\tightlist}{%
  \setlength{\itemsep}{0pt}\setlength{\parskip}{0pt}}
\setcounter{secnumdepth}{5}
\usepackage{indentfirst}
\parindent = 1em
\usepackage{dcolumn}
\newcolumntype{.}{D{.}{.}{-1}}
\usepackage{caption}
\captionsetup[table]{name=表}
\captionsetup[figure]{name=図}
\usepackage{hyperref}
\pagestyle{empty}
\usepackage{multicol}
\usepackage{ascmac}
\setpagelayout*{top=10truemm,bottom=30truemm,left=10truemm,right=10truemm}
\usepackage{tikz}
\usetikzlibrary{arrows.meta,decorations,decorations.pathreplacing,arrows,calc}
\usepackage{tabstackengine}
\usepackage{xcolor}
\usepackage{rotating}
\usepackage{txfonts}
\usepackage{fancybox}
\usepackage{dashbox}
\usepackage{tcolorbox}
\tcbuselibrary{theorems,skins}
\usepackage{siunitx}
\usepackage{framed}
\usepackage{enumerate}
\usepackage{lastpage}
\usepackage{pgfplots}
\pgfplotsset{compat=1.15}
\usepackage{mathrsfs}
\ifluatex
  \usepackage{selnolig}  % disable illegal ligatures
\fi

\author{}
\date{\vspace{-2.5em}}

\begin{document}

\renewcommand{\thefootnote}{}
\newcounter{kaunta}
\renewcommand{\thekaunta}{\arabic{kaunta}}
\newcommand{\kaunta}{\refstepcounter{kaunta}%
\thekaunta}
\def\question{\noindent\fbox{\large\makebox[1em]{\text{\kaunta}}} \hspace{1pt}}
\newcounter{skaunta}
\renewcommand{\theskaunta}{\arabic{skaunta}}
\newcommand{\skaunta}{\refstepcounter{skaunta}%
\theskaunta}
\def\squestion{(\text{\skaunta})\hspace{2.5pt}}
\newcommand{\maru}[1]{\raise0.2ex\hbox{\textcircled{\scriptsize{#1}}}}
\newcommand{\jsim}{\mathrel{\text{∽}}}
\newcommand{\jpara}{/\!/}

\newcounter{kcounter}
\setcounter{kcounter}{0}
\newcommand{\kana}{\refstepcounter{kcounter}\ifthenelse{\value{kcounter}=1}{ア}{\ifthenelse{\value{kcounter}=2}{イ}{\ifthenelse{\value{kcounter}=3}{ウ}{\ifthenelse{\value{kcounter}=4}{エ}{\ifthenelse{\value{kcounter}=5}{オ} {\ifthenelse{\value{kcounter}=6}{カ}{\ifthenelse{\value{kcounter}=7}{キ}{\ifthenelse{\value{kcounter}=8}{ク}{\ifthenelse{\value{kcounter}=9}{ケ}{\ifthenelse{\value{kcounter}=10}{コ}{\ifthenelse{\value{kcounter}=11}{サ}{\ifthenelse{\value{kcounter}=12}{シ}{\ifthenelse{\value{kcounter}=13}{ス}{\ifthenelse{\value{kcounter}=14}{セ}{\ifthenelse{\value{kcounter}=15}{ソ}{\ifthenelse{\value{kcounter}=16}{タ}{\ifthenelse{\value{kcounter}=17}{チ}{\ifthenelse{\value{kcounter}=18}{ツ}{\ifthenelse{\value{kcounter}=19}{テ}{\ifthenelse{\value{kcounter}=20}{ト}{\ifthenelse{\value{kcounter}=21}{ナ}{\ifthenelse{\value{kcounter}=22}{二}{\ifthenelse{\value{kcounter}=23}{ヌ}{\ifthenelse{\value{kcounter}=24}{ネ}{\ifthenelse{\value{kcounter}=25}{ノ}{\ifthenelse{\value{kcounter}=26}{ハ}{\ifthenelse{\value{kcounter}=27}{ヒ}{\ifthenelse{\value{kcounter}=28}{フ}{\ifthenelse{\value{kcounter}=29}{ヘ}{\ifthenelse{\value{kcounter}=30}{ホ}{\ifthenelse{\value{kcounter}=31}{マ}{\ifthenelse{\value{kcounter}=32}{ミ}{\ifthenelse{\value{kcounter}=33}{ム}{\ifthenelse{\value{kcounter}=34}{メ}{\ifthenelse{\value{kcounter}=35}{モ}{\ifthenelse{\value{kcounter}=36}{ヤ}{\ifthenelse{\value{kcounter}=37}{ユ}{\ifthenelse{\value{kcounter}=38}{ヨ}{\ifthenelse{\value{kcounter}=39}{ラ}{\ifthenelse{\value{kcounter}=40}{リ}{\ifthenelse{\value{kcounter}=41}{ル}{\ifthenelse{\value{kcounter}=42}{レ}{\ifthenelse{\value{kcounter}=43}{ロ}{\ifthenelse{\value{kcounter}=44}{ワ}{・}}}}}}}}}}}}}}}}}}}}}}}}}}}}}}}}}}}}}}}}}}}}}

\newcommand{\kuran}[1]{\framebox[1.5cm][c]{\maru{#1}}}

\newcommand{\degre}{\ensuremath{^\circ}}

\newcommand{\myarc}[1]{
   \tikz [baseline = (N.base), every node/.style={}] {
      \node [inner sep = 0pt] (N) {$\mathrm{#1}$};
      \draw [line width = 0.4pt] plot [smooth, tension=1.3] coordinates {
         ($(N.north west) + (0.1ex,0)$)
         ($(N.north)      + (0,0.5ex)$)
         ($(N.north east) + (0,0)$)
      };
   }
}

\makeatletter
\newenvironment{figurehere}{\def\@captype{figure}}{}
\makeatother

\newcommand{\haiten}[1]{%
\begin{flushright}%
\footnotesize{<#1>}%
\end{flushright}%
}

\newcommand{\goku}[1]{\fbox{\phantom{\text{#1}} \quad}}

\newgeometry{top=10truemm,bottom=10truemm,left=20truemm,right=20truemm}

\noindent\fbox{\large\makebox[1em]{\text{\refstepcounter{kaunta}%
\arabic{kaunta}}}} \hspace{1pt} 次の空欄に当てはまる言葉を書きなさい。

2点A,
Bを通り、両方にかぎりなくのびているものを\fbox{\phantom{\text{直線AB}} \quad}という。直線ABのうち、AからBまでの部分を\fbox{\phantom{\text{線分AB}} \quad}という。また、線分ABをBのほうへまっすぐにかぎりなくのばしたものを\fbox{\phantom{\text{半直線AB}} \quad}という。

図形を、一定の方向に、一定の距離だけ動かす移動を\fbox{\phantom{\text{平行移動}} \quad}という。線分ABと線分CDの長さが等しいことをAB
\fbox{\phantom{\text{$=$}} \quad}
CDとかく。図形をある点を中心として、一定の角度だけ回転させる移動を\fbox{\phantom{\text{回転移動}} \quad}といい、中心とする点を\fbox{\phantom{\text{回転の中心}} \quad}という。図形を\(\ang{180}\)だけ回転移動させることを\fbox{\phantom{\text{点対称移動}} \quad}という。また、図形をある直線を折り目として折り返す移動を\fbox{\phantom{\text{対称移動}} \quad}といい、折り目の直線を\fbox{\phantom{\text{対称の軸}} \quad}という。

平行な2直線を\fbox{\phantom{\text{平行線}} \quad}という。2直線AB,
CDが平行であることを

\noindent AB \fbox{\phantom{\text{$/\!/$}} \quad}
CDとかく。1つの点Oから出る2つの半直線OA,
OBによって角ができる。この角を\fbox{\phantom{\text{$\angle$}} \quad}AOBとかく。2直線が垂直であるとき、一方の直線を他方の直線の\fbox{\phantom{\text{垂線}} \quad}という。線分ABと直線\(l\)が垂直であることを、AB\fbox{\phantom{\text{$\perp$}} \quad}\(l\)とかく。

線分を2等分する点を、その線分の\fbox{\phantom{\text{中点}} \quad}という。線分の中点を通り、その線分に垂直な直線をその線分の\fbox{\phantom{\text{垂直二等分線}} \quad}という。2点からの距離が等しい点は、垂直二等分線上にある。

1つの角を2等分する半直線を、その角の\fbox{\phantom{\text{二等分線}} \quad}という。角の二等分線は、その角の\fbox{\phantom{\text{対称の軸}} \quad}である。角の二等分線上の点から角の二辺までの距離は等しい。また、角の内部にあって、その角の2辺までの距離が等しい点は、その角の二等分線上にある。

円周上の2点をA,
Bとするとき、AからBまでの円周の部分を弧ABといい、\fbox{\phantom{\text{$
   \tikz [baseline = (N.base), every node/.style={}] {
      \node [inner sep = 0pt] (N) {$\mathrm{AB}$};
      \draw [line width = 0.4pt] plot [smooth, tension=1.3] coordinates {
         ($(N.north west) + (0.1ex,0)$)
         ($(N.north)      + (0,0.5ex)$)
         ($(N.north east) + (0,0)$)
      };
   }
$}} \quad}と表す。円周上の2点を結ぶ線分を弦といい、両端がA,
Bである弦を\fbox{\phantom{\text{弦AB}} \quad}という。

円の中心を通る直線に垂直な直線を平行移動させていくと、1点だけで円と出あう場所がある。このとき、この直線は円に\fbox{\phantom{\text{接する}} \quad}といい、この直線を円の\fbox{\phantom{\text{接線}} \quad}、円と直線が接する点を\fbox{\phantom{\text{接点}} \quad}という。円の接線は、接点を通る半径に\fbox{\phantom{\text{垂直}} \quad}である。

弧の両端を通る2つの半径とその弧で囲まれた図形を\fbox{\phantom{\text{おうぎ形}} \quad}という。おうぎ形で、2つの半径のつくる角を\fbox{\phantom{\text{中心角}} \quad}という。

\def\@captype{figure}
\includegraphics{media_ku/image4.png}

上の\(\raise 0.2ex\hbox{\textcircled{\scriptsize{ア}}} \sim \raise 0.2ex\hbox{\textcircled{\scriptsize{エ}}}\)の立体のように、平面だけで囲まれた立体を\fbox{\phantom{\text{多面体}} \quad}という。また、\(\raise 0.2ex\hbox{\textcircled{\scriptsize{イ}}}\)や\(\raise 0.2ex\hbox{\textcircled{\scriptsize{エ}}}\)のような立体を\fbox{\phantom{\text{角錐}} \quad}という。\(\raise 0.2ex\hbox{\textcircled{\scriptsize{カ}}}\)のような立体を\fbox{\phantom{\text{円錐}} \quad}という。

底面が正三角形、正方形、\(\cdots\)で、側面がすべて合同な長方形である角柱をそれぞれ、正三角柱、正四角柱、\(\cdots\)という。また、底面が正三角形、正方形、\(\cdots\)で、側面がすべて合同な二等辺三角形である角錐を、それぞれ正三角錐、正四角錐、\(\cdots\)という。

立方体のように、多面体で次の2つの性質をもち、へこみのないものを\fbox{\phantom{\text{正多面体}} \quad}という。

\setcounter{kaunta}{0}
\begin{itemize}
\item[\noindent\fbox{\large\makebox[1em]{\text{\refstepcounter{kaunta}%
\arabic{kaunta}}}} \hspace{1pt}] どの面もすべて合同な正多角形である。
\item[\noindent\fbox{\large\makebox[1em]{\text{\refstepcounter{kaunta}%
\arabic{kaunta}}}} \hspace{1pt}] どの頂点にも面が同じ数だけ集まっている。
\end{itemize}

\end{document}
